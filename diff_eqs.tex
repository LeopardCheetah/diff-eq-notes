\documentclass{article}
\usepackage{amsfonts, amsmath, amssymb, amsthm}
\usepackage{relsize}
\usepackage{thmtools}

\usepackage{geometry}

\geometry{
    left = 0.75in, 
    right = 0.75in,
    top = 0.75in,
    bottom = 0.75in, 
    footskip = 0.25in,
}

\setlength{\parindent}{0pt} % no more stupid indentation

\usepackage{theoremref}
\usepackage[unicode]{hyperref}
\hypersetup{
    colorlinks = true,
    filecolor = black,
    linkcolor = blue,
    linktoc = all,
    urlcolor = black,
    pdftitle = {Differential Equation Notes} % TO CHANGE LATER 
}


% for colorings.
\usepackage[dvipsnames]{xcolor}
\usepackage{tcolorbox}


\usepackage{subfiles} % Best loaded last in the preamble

%%%%%%%%%%%%%%%%%%%%%%%%%%
% horizontal bar
% \newcommand{\hr}[1]{\rule{\linewidth}{#1}}
\newcommand{\hr}{\rule{\linewidth}{1pt}}
\newcommand{\zz}{\hr} % \hr will go to \hat so this is to prevent that
\newcommand{\spacer}{\vspace{2mm} \hr \vspace{5mm}}
\newcommand{\emspacer}{\phantom \\} % empty spacer
%%%%%%%%%%%%%%%%%%%%%%%%%%




\theoremstyle{definition}
\newtheorem*{definition}{Definition}

%%%%%%%%%%%%%%%%%%%%%%%%%%%%%%%%%%%%%
% from latex overflow
% https://tex.stackexchange.com/questions/53978/custom-theorem-numbering
\newtheorem{innercustomthm}{Exercise}
\newenvironment{exercise}[1]{\renewcommand\theinnercustomthm{#1}\innercustomthm \phantom \\ }{\endinnercustomthm}

% taken from analysis
\tcolorboxenvironment{exercise}{colframe=Orange, colback=White, after skip=10pt}
\AtEndEnvironment{exercise}{\qed}
%%%%%%%%%%%%%%%%%%%%%%%%%%%%%%%%%%%%%%%%%%%%%%%%%%%%%%%%%%%%

%%%%%%%%%%%%%%%%%%%%%%%%%%%%%%%%

\title{Differential Equations Notes}
\author{Alex Z}
\date{Fall 2025}

\begin{document}

\maketitle
\newpage

\tableofcontents
\newpage

%--------------------------------------------------- end header ----------------------------

\section*{Remarks}

This notes thing was started on 10/19/2025.

I am to work through the WHOLE of the textbook (Elementary Differential Equations and Boundary Value Problems (11th ed)) by the end of this quarter (12/13/2025).

Hopefully, I'll also get around ~20-40\% of the problems in the textbook done.

\newpage

\section{Introduction}

aka chapter 1

\subsection{Introduction for the Introduction}
\subfile{chapters/ch1/1-1.tex} 

\spacer

\subsection{Introduction to Solutions}
\subfile{chapters/ch1/1-2.tex}

\spacer

\subsection{Classification of Diffy Qs}
\subfile{chapters/ch1/1-3.tex}


\newpage
%%%%%%%%%%%%%%%%%%%%%%%%%%%%%%%%%%%%%%%%%%%%%

\section{First-Order Diffy Qs}

aka chapter 2 

for chapter 2, all diffy qs will be first order.

\subsection{Linear ODEs: Method of Integrating Factors}
\subfile{chapters/ch2/2-1.tex}

\subsection{Separable Differential Equations}
\subfile{chapters/ch2/2-2.tex}



\end{document}