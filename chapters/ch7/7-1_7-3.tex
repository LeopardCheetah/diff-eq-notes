\documentclass[../../diff_eqs.tex]{subfiles}

\begin{document}

%%%%%%%%%%%%%%%%%%%%%%%%%%%%%%%%%%%%%%%%%%%%%%%%%%%%%%%%%%%%
% docs/syntax:

% definitions
% \begin{definition}[Definition]
%     Definition 1
% \end{definition}

% hr
% \hr

% exercise
% \begin{exercise}{problem number}
%
%    problem starts
% \end{exercise}

%%%%%%%%%%%%%%%%%%%%%%%%%%%%%%%%%%%%%%%%%%%%%%%%%%%%%%%%%%%%

% 7-1 - 7.3 pdf

\subsection{Introduction}

Essentially, we consider systems of first-order equations since any higher order differential equation can inevitably be transformed into multiple first order linear transformations.

\vspace{0.2cm}

Moreover, for any $y^{(n)} = F(t, y, y', \dots, y^{(n-1)})$, we cna make the substituions $x_1 = y$, $x_2 = y'$, $\dots$, $x_n = y^{(n - 1)}$ and thus eventually find $x_1' = F_1(t, x_1, x_2, \dots, x_n)$, $x_2' = F_2(t, x_1, x_2, \dots, x_n)$ and so on. Thus, we have effectively converted a general differential equation into many teeny tiny first-order differential equations (that are each in their own way, granted, hard to solve).


\subsection{Matrices}
(note: all uppercase letters from here on out ($A$, $B$, $C$, $\dots$) will most likely represent matrices from here on out unless they are in function notation (e.g. $F(t)$ would be a function)). 

\vspace{0.2cm}

Various matrix preliminaries are covered here. Do note that when the book talks about the \textbf{adjoint} of $A$, they mean the \textbf{transpose of the conjugate matrix of $A$} rather than the cofactor expansion matrix of $A$. 

\vspace{0.2cm}

Integrals, derivatives, and [x] over matrices of functions are just those same operations applied to each individual operations (boring). For example, $\mathlarger{\int A \ dt = \int a_{ij} \ dt}$.



\subsection{More Linear Algebra}

(This is just a review of Math 4a......)


\end{document}
