\documentclass[../../diff_eqs.tex]{subfiles}

\begin{document}

%%%%%%%%%%%%%%%%%%%%%%%%%%%%%%%%%%%%%%%%%%%%%%%%%%%%%%%%%%%%
% docs/syntax:

% definitions
% \begin{definition}[Definition]
%     Definition 1
% \end{definition}

% hr
% \hr

% exercise
% \begin{exercise}{problem number}
%
%    problem starts
% \end{exercise}

%%%%%%%%%%%%%%%%%%%%%%%%%%%%%%%%%%%%%%%%%%%%%%%%%%%%%%%%%%%%


So what happens if there is an eigenvalues of $\matr{A}$ that has a multiplicity $m > 1$? Long story short, we basically use the variation of parameters method with $e^{rt}$ and $te^{rt}$ to find our answer. 

\vspace{0.2cm}

In more detail, assume that there is an eigenvalue $\lambda = \lambda_m$ such that the multiplicity of $\lambda_m$ is $m$ (e.g. $(\lambda_m - \lambda)^m$ is in the characteristic polynomial of $\matr{A} - \lambda\matr{I}$). In this case, one trivial solution is simply $\matr{x} = \xi e^{\lambda_m t}$. Assuming no other linearly independent eigenvectors can be found, to find the second solution, we let $\matr{x} = \alpha_1 te^{\lambda_m t} + \alpha_2 e^{\lambda_m t}$, plug this particular solution into the differential equation, and simplify to find what $\alpha_1$ and $\alpha_2$ are. We then repeat this process (e.g. assume $\matr{x} = \alpha_1 \frac{t^n}{n!}e^{\lambda_m t} + \alpha_2 \frac{t^{n - 1}}{(n - 1)!}e^{\lambda_m t} + \cdots + \alpha_{n + 1}e^{\lambda_m t}$) until we get $m$ linearly independent solutions for $\matr{x}$. Note that your various solutions for $\matr{x}$ along the way will include previous derived solutions (e.g. $\matr{x} = \begin{pmatrix}
    2 \\ 1 \\ 0
\end{pmatrix}\frac{t^2}{2} e^t + \begin{pmatrix}
    100 \\ -32 \\ -101
\end{pmatrix}te^t + \begin{pmatrix}
    \pi \\ \zeta(3) \\ e^{e^{\gamma}}
\end{pmatrix}e^t$).

\vspace{0.2cm}

(Some stuff about Jordan Forms and Fundamental Matrices discussed in Section 7.7 are brought up for discussion again here.)

% do 4-5, 6a-8a, 9a, 10a, 14, 15(?!), 17, 18 (?!)
\begin{exercise}{4}
    
    4. The characteristic equation is $(\lambda - 2)^2(\lambda + 1) = 0$ which leads to the eigenvectors $\xi = \begin{bmatrix}
        -3 \\ 4 \\ 2 
    \end{bmatrix}, \begin{bmatrix}
        0 \\ 1 \\ -1
    \end{bmatrix}$ with the latter $\xi$ coming from the eigenvalue $\lambda = 2$. 

    To find the extra solution then, we let $\matr{x} = \alpha te^{2t} + \beta e^{2t}$ and find 
    $$2\beta e^{2t} + \alpha e^{2t} + 2\alpha t e^{2t} = \matr{A}(\alpha te^{2t} + \beta e^{2t})\text{.}$$

    Comparing coefficients, we come to the conclusion that 
    $2 \alpha = \matr{A}\alpha$ and $2\beta + \alpha = \matr{A}\beta$. The first equation ($(\matr{A} - 2I)\alpha = \matr{0}$) is already solved for us as that was the eigenvalue we found before, so we then solve $(\matr{A} - 2I)\beta = \alpha$, finding the general solution to be $\beta = \begin{bmatrix}
        1 \\ 0 \\ 1
    \end{bmatrix} - a\begin{bmatrix}
        0 \\ 1 \\ -1
    \end{bmatrix}$. 

    As such, our final solution is 
    $$\matr{x} = \begin{bmatrix}
        0 \\ 1 \\ -1
    \end{bmatrix}te^{2t} + \begin{bmatrix}
        1 \\ 0 \\ 1 
    \end{bmatrix}e^{2t} - a\begin{bmatrix}
        0 \\ 1 \\ -1
    \end{bmatrix}e^{-2t} = \begin{bmatrix}
        0 \\ 1 \\ -1
    \end{bmatrix}te^{2t} + \begin{bmatrix}
        1 \\ 0 \\ 1 
    \end{bmatrix}e^{2t}$$

    since the third term in the solution is already covered by our general solution. 
    As such, the general solution for the system of equations is 

    $$\matr{x} = c_1\begin{bmatrix}
        -3 \\ 4 \\ 2
    \end{bmatrix}e^{-t} + c_2\begin{bmatrix}
        0 \\ 1 \\ -1
    \end{bmatrix}e^{2t} + c_3e^{2t}\begin{bmatrix}
        1 \\ t \\ 1 - t
    \end{bmatrix}\text{.}$$
\end{exercise}

\begin{exercise}{5}
    
    Following the same process as in problem 4, we find the characteristic polynomial to be $(\lambda - 2)(\lambda + 1)^2 = 0$, and the eigenvectors to be $\begin{bmatrix}
        1 \\ 1 \\ 1
    \end{bmatrix}$ for $\lambda = 2$ and $\begin{bmatrix}
        1 \\ -1 \\ 0
    \end{bmatrix}, \begin{bmatrix}
        1 \\ 0 \\ -1
    \end{bmatrix}$ to be the eigenvectors for $\lambda = -1$. Since the eigenvectors for the repeated eigenvalue we have found are linearly independent, we can proceed directly to the solution and conclude that 
    $$\matr{x} = c_1\begin{bmatrix}
        1 \\ 1 \\ 1
    \end{bmatrix}e^{2t} + c_2\begin{bmatrix}
        1 \\ -1 \\ 0
    \end{bmatrix}e^{-t} + c_3\begin{bmatrix}
        1 \\ 0 \\ -1
    \end{bmatrix}e^{-t}\text{.}$$
\end{exercise}



\begin{exercise}{6a-10a}
    
    6a. $\matr{x} = 2\begin{bmatrix}
        1 \\ 1
    \end{bmatrix}e^{-3t} + 4\begin{bmatrix}
        \frac{1}{4} + t \\ t 
    \end{bmatrix}e^{-3t} = \begin{bmatrix}
        3 + 4t \\ 2 + 4t
    \end{bmatrix}e^{-3t}$. 
    
    7a. $\matr{x} = -\begin{bmatrix}
        1 \\ 1
    \end{bmatrix}e^{-t} -6\begin{bmatrix}
        -\frac{2}{3} + t \\ t 
    \end{bmatrix}e^{-t} = \begin{bmatrix}
        3 + 6t \\ -1 + 6t
    \end{bmatrix}e^{-t}$. 

    8a. $\matr{x} = 4\begin{bmatrix}
        -3 \\ 1
    \end{bmatrix} -14 \begin{bmatrix}
        -1 - 3t \\ t 
    \end{bmatrix} = \begin{bmatrix}
        2 + 42t \\ 4 - 14t
    \end{bmatrix}$.

    \vspace{0.25cm}

    9a. $\matr{x} = 3\begin{bmatrix}
        0 \\ 0 \\ 1
    \end{bmatrix}e^{2t} + 2\begin{bmatrix}
        0 \\ 1 \\ -6
    \end{bmatrix}e^t + 4\begin{bmatrix}
        -\frac{1}{4} \\ t \\ -\frac{21}{4} - 6t
    \end{bmatrix}e^t = \begin{bmatrix}
        0 \\ 0 \\ 3
    \end{bmatrix}e^{2t} + \begin{bmatrix}
        -1 \\ 2 + 4t \\ -33 - 24t
    \end{bmatrix}e^t$.

    \vspace{0.25cm}

    10a. $\matr{x} = \mathlarger{\frac{4}{3}\begin{bmatrix}
        1 \\ 1 \\ 1
    \end{bmatrix}e^{-\frac{t}{2}} - \frac{5}{3}\begin{bmatrix}
        1 \\ -1 \\ 0
    \end{bmatrix}e^{-\frac{7t}{2}} + \frac{7}{3}\begin{bmatrix}
        1 \\ 0 \\ -1 
    \end{bmatrix}e^{-\frac{7t}{2}}}$.

\end{exercise}

\begin{exercise}{14}
    
    14a. Trivial; the discriminant is literally $L - 4R^2C$ so the result follows. \\ 
    14b. The repeated root is $\lambda = -\frac{1}{2}$, and the solution is 
    $$\begin{bmatrix}
        I \\ V
    \end{bmatrix} = -\begin{bmatrix}
        1 \\ -2 
    \end{bmatrix}e^{-\frac{t}{2}} + \begin{bmatrix}
        2 + t \\ -2t 
    \end{bmatrix}e^{-\frac{t}{2}} = \begin{bmatrix}
        1 + t \\ 2 - 2t
    \end{bmatrix}e^{-\frac{t}{2}}\text{.}$$
\end{exercise}

\begin{exercise}{15}
    
    15a. $(\matr{A} - 2\matr{I})((\matr{A} - 2\matr{I})\eta) = \matr{0} \rightarrow (\matr{A} - 2\matr{I})^2\eta = \matr{0}$. \\ 
    15b. In this case, we can manually verify that $\begin{pmatrix}
        -1 & -1 \\ 1 & 1
    \end{pmatrix}^2 = \matr{0}$. \\ 
    15c. Simple matrix multiplication shows $\xi = (1, -1)^T$. \\ 
    15d. More matrix multiplication shows $\xi = (-1, 1)^T$. \\ 
    15e. As long as $k_1 \not = -k_2$, $\xi$ and $\eta$ will be independent; $\xi = (-k_1 - k_2, k_1 + k_2)^T$. $\xi$ in this case will be a multiple of $\xi^{(1)}$.
\end{exercise}

\begin{exercise}{17a-d}
    
    17a. The characteristic polynomial ends up being $-\lambda^3 + 6\lambda^2 - 12\lambda + 8 = -(\lambda - 2)^3 = 0$ which yields an eigenvalue $2$ of multiplicity $3$. Going eigenvector hunting reveals that indeed, $\xi = (0, 1, -1)^T$ is the only eigenvector. \\ 
    17b. $\matr{x}^{(1)} = \xi e^{2t}$. 

    \vspace{0.2cm}

    17c. I mean they just kinda do satisfy those equations. Solving for $\eta$ and neglecting the part we already found, we find $\eta = (1, 1, 0)^T$, meaning our second solution is literally $\matr{x}^{(2)} = \xi te^{2t} + \eta e^{2t}$ for the $\xi$ and $\eta$ we have found. 

    \vspace{0.2cm}

    17d. One particular solution for $\zeta$ is $\zeta = (-2, 3, 0)^T$, which means our final general solution can be written as 
    
    $$\matr{x} = c_1\begin{pmatrix}
        0 \\ 1 \\ -1
    \end{pmatrix}
    e^{2t} + c_2\left(t\begin{pmatrix}
        0 \\ 1 \\ -1
    \end{pmatrix} + \begin{pmatrix}
        1 \\ 1 \\ 0
    \end{pmatrix}\right)e^{2t} + c_3\left(\frac{t^2}{2}\begin{pmatrix}
        0 \\ 1 \\ -1
    \end{pmatrix} + t\begin{pmatrix}
        1 \\ 1 \\ 0
    \end{pmatrix} + \begin{pmatrix}
        -2 \\ 3 \\ 0
    \end{pmatrix}\right)e^{2t}\text{.}$$
\end{exercise}

\end{document}
