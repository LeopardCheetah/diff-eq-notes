\documentclass[../../diff_eqs.tex]{subfiles}

\begin{document}

%%%%%%%%%%%%%%%%%%%%%%%%%%%%%%%%%%%%%%%%%%%%%%%%%%%%%%%%%%%%
% docs/syntax:

% definitions
% \begin{definition}[Definition]
%     Definition 1
% \end{definition}

% hr
% \hr

% exercise
% \begin{exercise}{problem number}
%
%    problem starts
% \end{exercise}

%%%%%%%%%%%%%%%%%%%%%%%%%%%%%%%%%%%%%%%%%%%%%%%%%%%%%%%%%%%%


This subsection focuses on equations of the form 

$$\matr{x'} = \matr{A}\matr{x}$$ 

where $\matr{A}$ is a $n \times n$ matrix of real-valued constants.

\vspace{0.2cm}

Assuming\footnote{Some stuff about a phase portrait/plane is talked about here although those tools are primarily used for visualization purposes.} $\matr{x} = \xi e^{rt}$, after substituting it into the above equation, we eventually derive the equation 

$$(\matr{A} - r\matr{I})\xi = 0\text{,}$$

which means solutions to $\matr{x}$ are given pairs of eigenvalue-eigenvector combinations $(r, \xi)$. When $\matr{A}$ is specifically a $2 \times 2$ matrix, if the eigenvalues of $\matr{A}$ have opposite signs, then the origin is a saddle point and an unstable equilibrium. If on the other hand, the eigenvalues of $\matr{A}$ have the same sign, then the origin is a \textbf{node} and $\matr{0}$ is a stable equlibrium if the eigenvalues are negative and unstable if the eigenvalues are positive.

\vspace{0.2cm}

Returning to the more general case of when $\matr{A}$ is a $n \times n$ matrix, the eigenvalues of $\matr{A}$ ($r_1$, $r_2$, $\dots$, $r_n$) can either be 
\begin{enumerate}
    \item all real and different from one another,
    \item some eigenvalues are complex conjugate pairs of each other, or 
    \item some eigenvalues are repeated.
\end{enumerate}

The first case is easy to take care of; if all $n$ eigenvalues are real and different, then their corresponding eigenvectors ($\xi^{(i)}$) will all be linearly independent and as such $\matr{x} = c_1\xi^{(1)}e^{r_1t} + \cdots + c_n\xi^{(n)}e^{r_nt}$. Section 7.6 deals with case two, of when some eigenvalues are complex conjugate pairs of each other. Section 7.8 deals with the case of repeated eigenvalues.


\end{document}
