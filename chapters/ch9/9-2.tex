\documentclass[../../diff_eqs.tex]{subfiles}

\begin{document}

%%%%%%%%%%%%%%%%%%%%%%%%%%%%%%%%%%%%%%%%%%%%%%%%%%%%%%%%%%%%
% docs/syntax:

% definitions
% \begin{definition}[Definition]
%     Definition 1
% \end{definition}

% hr
% \hr

% exercise
% \begin{exercise}{problem number}
%
%    problem starts
% \end{exercise}

%%%%%%%%%%%%%%%%%%%%%%%%%%%%%%%%%%%%%%%%%%%%%%%%%%%%%%%%%%%%

In this section we will be concerned with Autonomous systems of two functions of the form

$$\frac{dx}{dt} = F(x, y), \ \ \frac{dy}{dt} = G(x, y)\text{.}$$

with initial condition $x(t_0) = x_0$, $y(t_0) = y_0$. This system can also be written in matrix form as 

$$\frac{d\matr{x}}{dt} = \matr{f}(\matr{x}), \ \matr{x}(t_0) = \matr{x}^0$$

where $\matr{x} = (x, y)^T = (x(t), y(t))^T$, $\matr{f}(\matr{x}) = (F(x, y), G(x, y))^T$, and $\matr{x}^0 = (x_0, y_0)^T$.

For simplicity and so theorems hold, we will assume $F$ and $G$ are continuous and their partial derivatives are also continuous.\footnote{Rigorously speaking, their partial derivatives only need to be continuous over some domain $D$ of the $xy$-plane. But since most functions we work with are very simple, it might as well be (almost) the whole $xy$-plane over which the partial derivatives are continuous.} 
\newpage % needed here to fix the begin definition function declaration

\begin{definition}[Autonomous]
    A differential equation system is said to be \textbf{autonomous} if the systems do not depend on time. In particular, the system given above ($\frac{dx}{dt} = F(x, y), \ \ \frac{dy}{dt} = G(x, y)$) is autonomous, and so is the system $\matr{x'} = \matr{A}\matr{x}$ as long as all terms in $\matr{A}$ do not involve the independent variable $t$.    
\end{definition}

The distinction between autonomous and nonautonomous systems is important as the condition of autonomy guarantees that there is only one trajectory crossing through the point $(x_0$, $y_0)$ regardless of time. 

\subsubsection{Stability and Instability}

For autonomous systems of the form $$\matr{x'} = \matr{f}(\matr{x})\text{,}$$

\begin{definition}[Critical Points]
    \textbf{Critical points} are defined as points to the above system where $\matr{f}(\matr{x}) = \matr{0}$. Since $\matr{x'} = \matr{0}$, critical points must be constant solutions to the system.
\end{definition}

\begin{definition}[Stability]
    A critical point of the solution $\matr{x}^{\circ}$ is said to be \textbf{stable} if for any $\varepsilon > 0$, there exists $\delta > 0$ such that every solution $\matr{x}$ which at $t = 0$ satisfies 
    $$||\matr{x}(0) - \matr{x}^{\circ}|| < \delta$$ 

    exists for all $t > 0$ and satisfies 
    $$||\matr{x}(t) - \matr{x}^{\circ}|| < \varepsilon$$
    
    for all $t \geq 0$.    
\end{definition}

Essentially, this definition codifies the notion that a given solution $\matr{x}^{*}$ should stay bounded within the critical point. Note however that this definition of stability does not require $\matr{x}^{*}$ to converge to $\matr{x}^{\circ}$; instead, it merely requires that $\matr{x}^{*}$ not leave an open disk\footnote{If $\matr{x}$ is $n$-dimensional, then the statement should be amended to read ``\dots not leave an open $n-1$-sphere of radius \dots''.} of radius $\varepsilon$ centered at $\matr{x}^{\circ}$ for all $t \geq 0$. 

\vspace{0.2cm}

Any critical points for which the condition of stability doesn't hold are said to be \textbf{unstable}.

\vspace{0.2cm}

The following definition thus distinguishes between stability and asymptotic stability:

\begin{definition}[Asymptotic Stability]
    A critical point $\matr{x}^{\circ}$ is said to be \textbf{asymptotically stable} if it is stable and there exists a $\delta_0 > 0$ such that if a solution $\matr{x} = \matr{x}(t)$ satisfies 
    $$||\matr{x}(0) - \matr{x}^{\circ}|| < \delta_0\text{,}$$
    $$\text{then } \lim_{t \to \infty} \matr{x}(t) = \matr{x}^{\circ}\text{.}$$

    In english, if a trajectory starts ``sufficiently close'' to $\matr{x}^{\circ}$ (within a $\delta_0$), then it must eventually approach $\matr{x}^{\circ}$ as $t \to \infty$.
\end{definition}

\hr


\begin{definition}[Basin of Attraction]
    For a two-dimensional (potentially non-linear) autonomous system with at least one asymptotically critically point, we define the \textbf{basin of attraction} for a critical points to be the set of all points $P$ such that a trajectory passing through $P$ eventually converges to said critical point as $t \to \infty$. 

    If there is a boundary to a \textbf{basin of attraction}, that trajectory which bounds the basin is called a \textbf{separatrix} as it seperates the trajectories that converge and the trajectories that don't.
\end{definition}

If we're lucky, we can determine trajectories of a two-dimensional autonomous system by solving just a first-order differential equation. Namely, since $F(x, y)$ and $G(x, y)$ don't depend on $t$, we have 

$$\frac{dy}{dx} = \frac{dy/dt}{dx/dt} = \frac{G(x, y)}{F(x, y)}$$

which is a first-order differential equation. In general, the differential equation arising from the quotient $\frac{G(x, y)}{F(x, y)}$ may not be solvable.



\begin{exercise}{14a-20a}

    For this exercise, I recommend using \href{https://choosedews.github.io/PhasePlane/}{this} or \href{https://homepages.bluffton.edu/~nesterd/apps/slopefields.html}{this} online plotter to plot some nice looking solutions.

    14a. Following the equation above, we have $\frac{dy}{dx} = \frac{G(x, y)}{F(x, y)} = \frac{8x}{2y} \rightarrow H(x, y) = y^2 - 4x^2 = c$. Solutions to this system generally look like (one-directional) hyperbolas. \\
    15a. $\frac{dy}{dx} = \frac{-8x}{2y} \rightarrow H(x, y) = y^2 + 4x^2 = c$. As the equation suggests, solutions to this system look like an ellipse. 

    \vspace{0.2cm}

    16a. $\frac{dy}{dx} = \frac{2x + y}{y}$. Using the substitution $y = vx$ and $dy = vdx + xdv$, we thus have 

    $$\frac{vdx + xdv}{dx} = v + \frac{dv}{dx}x = \frac{2 + v}{v} \rightarrow \frac{dv}{dx}x = \frac{2 + v - v^2}{v}$$ 
    
    which with partial fraction decomposition and a bunch of tedious integration simplification reveals $H(x, y) = (x + y)(y - 2x)^2 = c$.

    17a. $\frac{dy}{dx} = \frac{x + y}{x - y}$. Using the substitution $y = vx$ again, we thus have $H(x, y) = \arctan\left(\frac{y}{x}\right) - \ln\left(\sqrt{\frac{y^2}{x^2} + 1}\right) - \ln x = \arctan\left(\frac{y}{x}\right) - \ln\sqrt{x^2 + y^2} = c$.

    18a. While this equation looks super complicated, if you rearrange it into the form $(2xy^2 - 6xy) + (2x^2y - 3x^2 - 4y)y' = 0$, you quickly find the differential equation is exact so $H = x^2y^2 - 3x^2y - 4y^2 = c$.

    19a. $\frac{dy}{dx} = \frac{-\sin x}{y} \rightarrow \frac{y^2}{2} - \cos(x) = c$.

    20a. $\frac{y^2}{2} + \frac{x^2}{2} - \frac{x^4}{24} = c$.
\end{exercise}

\end{document}
