\documentclass[../../diff_eqs.tex]{subfiles}

\begin{document}

%%%%%%%%%%%%%%%%%%%%%%%%%%%%%%%%%%%%%%%%%%%%%%%%%%%%%%%%%%%%
% docs/syntax:

% definitions
% \begin{definition}[Definition]
%     Definition 1
% \end{definition}

% hr
% \hr

% exercise
% \begin{exercise}{problem number}
%
%    problem starts
% \end{exercise}

%%%%%%%%%%%%%%%%%%%%%%%%%%%%%%%%%%%%%%%%%%%%%%%%%%%%%%%%%%%%

The following table/theorem (theorem 9.3.1 in the book) recaps the stability properties of the origin for the two dimensional linear system $\matr{x'} = \matr{A}\matr{x}$.

\begin{definition}[Stability of $\matr{0}$]
    Let the eigenvalues for $\matr{A}$ be $r_1$ and $r_2$. Then, the critical point $\matr{x} = \matr{0}$ is 
    \begin{enumerate}
        \item asymptotically stable if $r_1$ and $r_2$ have negative real part,
        \item stable if $r_1$ and $r_2$ have $0$ real part (e.g. $r_1$ and $r_2$ are pure imaginary eigenvalues),
        \item unstable if either $r_1$ and $r_2$ have any sort of positive real part.
    \end{enumerate}
\end{definition}

From the table we can conclude that small pertrubations in the roots $r_1$ and $r_2$ only really matter when $r_1$ and $r_2$ are pure imaginary eigenvalues, as any addition of a real part (either positive or negative) will cause the system to spiral inwards towards $\matr{0}$ or outwards to infinity. Thus, while $\matr{0}$ is stable if $r_1$ and $r_2$ have 0 real part, this stability is itself potentially unstable. 

\subsubsection{Linear Approximations}

\begin{definition}[Isolated critical point]
    We say that a critical point $\matr{x}^{\circ}$ is an \textbf{isolated critical point} of the system if there is some disk around $\matr{x}^{\circ}$ with radius $r > 0$ such that there exists no other critical poitns in that disk.     
\end{definition}

Considering the linearization of trajectories around the origin ($\matr{x} = \matr{0})$ for the non-linear system $\matr{x'} = \matr{A}\matr{x} + \matr{g}$, we first assume $\matr{0}$ is an isolated critical point of the system. We also assume that around the critical point (in this case $\matr{x} = \matr{0}$) that $\matr{g}$ is small, or in rigorous terms,

$$\lim_{\matr{x} \to \matr{0}} \frac{|\matr{g}|}{|\matr{x}|} = 0$$

assuming\footnote{It is also possible to convert this limit into polar coordinates ($x = r\cos\theta$, $y = r\sin\theta$, $|\matr{x}| = r$) to make this limit easier to evaluate.} $\matr{g}$ has continuous first partial derivatives. If the above condition is satisfied, the system we described above can then be called a \textbf{locally linear system}. 

\vspace{0.2cm}

Anyways, here's a cool theorem we can use to determine local linearity:

\begin{definition}[Theorem 9.3.2 (p. 410)]
    The system described by 
    $$\matr{x'} = \matr{f}(\matr{x}) = \begin{pmatrix}
        x' \\ y'
    \end{pmatrix} = \begin{pmatrix}
        F(x, y) \\ G(x, y)
    \end{pmatrix}$$
    is locally linear in the neighborhood of a critical point $\matr{x}^{\circ}$ whenever $F$ and $G$ have continuous partial derivatives up to order two (e.g. $F$ and $G$ are twice differentiable).
\end{definition}

If the above condition holds, then the nonlinear system near $\matr{x}^{\circ} = (x^{\circ}, y^{\circ})$ can be approximated by the linear system 

$$\frac{d}{dt}\begin{pmatrix}
    x - x^{\circ} \\ y - y^{\circ}
\end{pmatrix} = \begin{pmatrix}
    F_x(x^{\circ}, y^{\circ}) & F_y(x^{\circ}, y^{\circ}) \\ 
    G_x(x^{\circ}, y^{\circ}) & G_y(x^{\circ}, y^{\circ})
\end{pmatrix}\begin{pmatrix}
    x - x^{\circ} \\ y - y^{\circ}
\end{pmatrix} \ \rightarrow \ \frac{d\matr{u}}{dt} = \frac{d\matr{f}}{d\matr{x}}\left(\matr{x}^{\circ}\right)\matr{u}$$

where $\matr{u} = \begin{pmatrix}
    x - x^{\circ} \\ y - y^{\circ}
\end{pmatrix}$.



The general coefficient matrix in the above equation

$$\matr{J} = \matr{J}[F, G](x, y) = \begin{pmatrix}
    F_x(x, y) & F_y(x, y) \\ 
    G_x(x, y) & G_y(x, y)
\end{pmatrix}$$ 

is called the \textbf{Jacobian}\footnote{Recognize this from Calc 3?} of $F$ and $G$ with repsect to $x$ and $y$. For the linear approximation system above, we need to assume $\det \matr{J}(\matr{x}^{\circ}) \not = 0$ so that $\matr{x}^{\circ}$ is an isolated critical point in our linear approximation system.

Anyways, to relate the properties of stability of linear and locally linear systems, here's a theorem and a table from the textbook:

\begin{center}
    \includegraphics[scale=0.6]{9-3_table.png}
\end{center}

or in summary, except for two special cases ($r_1 = r_2$, $\lambda = 0$), the non-linear terms of the nonlinear system do not affect the stability of the system determined by the linear systems.

If every trajectory approaches the critical point at the origin, then the critical point $\matr{0}$ is said to be \textbf{globally asymptotically stable}.

\vspace{0.2cm}

/// The textbook then goes into more detail about the stability of a damped pendulum around some critical points (Page 413-415). ///

\hr

\begin{exercise}{1-3}

    1. We verify $(0, 0)$ is a critical point by noting that $(0, 0)$ satsifies $x - y^2 = x - 2y + x^2 = 0$. 

    To find the locally linear version of the system, since both $F = x - y^2$ and $G = x - 2y + x^2$ have continuous second partial derivatives, we can invoke theorem 9.3.2 which tells us that our non-linear system can be approximated by the linear system 

    $$\begin{pmatrix}
        x' \\ y' 
    \end{pmatrix} = \begin{pmatrix}
        F_x(0, 0) & F_y(0, 0) \\ G_x(0, 0) & G_y(0, 0)
    \end{pmatrix}\begin{pmatrix}
        x \\ y
    \end{pmatrix} = \begin{pmatrix}
        1 & 0 \\ 1 & -2
    \end{pmatrix}\begin{pmatrix}
        x \\ y
    \end{pmatrix}$$ 
    
    which is verifiably linear. This linear system has two real, distinct eigenvalues $\lambda = 1$ and $\lambda = -2$ so by Table 9.3.1, both the linear and non-linear system around $(0, 0)$ are unstable as in both cases, the origin is a saddle point.

    \vspace{0.2cm}

    2. $(0, 0)$ can be verified to be a critical point, and it should be noted that it is not the only critical point of the system (see $(-2, \pi)$, $(0, 2\pi)$, etc). Since the partial derivatives of both $F$ and $G$ up to order two are continuous, we have 

    $$\begin{pmatrix}
        x' \\ y'
    \end{pmatrix} = \begin{pmatrix}
        0 & 1 \\ -1 & 0
    \end{pmatrix}\begin{pmatrix}
        x \\ y
    \end{pmatrix}$$ 

    so $\lambda = \pm i$. As such, for the linear system, the origin is a stable center. For the nonlinear system however, the origin could be either a center or a spiral point so its stability is unknown.

    \vspace{0.2cm}

    3. $(0, 0)$ is verifiably a critical point, and the corresponding locally linear matrix is $\matr{A} = \begin{pmatrix}
        1 & 0 \\ 1 & 1
    \end{pmatrix}$ which yields the eigenvalue $\lambda = 1$ with multiplicity two. Thus, the origin in both the locally linear and nonlinear systems is unstable, with the origin being an improper node\footnote{Since the system only has one eigenvector, namely $(0, 1)^T$.} in the locally linear system and either a spiral point or a node in the nonlinear system.    
\end{exercise}

(For the exercises below, \href{https://homepages.bluffton.edu/~nesterd/apps/slopefields.html}{this} website gives some marvelous slope fields/phase diagrams.)

\begin{exercise}{4abc}
    
    4a. Critical points to the equations occur when $\frac{dx}{dt} = 0 = \frac{dy}{dt}$. Examining the first equation, $\frac{dx}{dt}$ is $0$ when $x = y$ or $x = -2$. The second equation shows $\frac{dy}{dx} = 0$ when $x = -y$ or $x = 4$. Testing out each of the 4 combination of critical equations between the points, we find that there are 3 critical points, namely $(0, 0)$, $(-2, 2)$, and $(4, 4)$.

    \vspace{0.2cm}

    4b. As a straightforward application of Theorem 9.3.2, for each critical point $(x^{\circ}, y^{\circ})$ found above, the locally linear system is of the form 

    $$\begin{pmatrix}
        x' \\ y' 
    \end{pmatrix} = \begin{pmatrix}
        y^{\circ} - 2x^{\circ} - 2 & x^{\circ} + 2 \\ 4 - y^{\circ} - 2x^{\circ} & 4 - x^{\circ}
    \end{pmatrix}\begin{pmatrix}
        x \\ y
    \end{pmatrix} \rightarrow \begin{pmatrix}
        -2 & 2 \\ 4 & 4
    \end{pmatrix}, \begin{pmatrix}
        0 & 4 \\ 6 & 6
    \end{pmatrix}, \begin{pmatrix}
        -6 & 6 \\ -8 & 0 
    \end{pmatrix}\text{.}$$ 
    

    \vspace{0.2cm}

    4c. For the system with critical point $(0, 0)$, its eigenvalues are $\lambda = 1 \pm \sqrt{17}$, which means that in the nonlinear system, the origin is an unstable saddle point. 

    For the system with critical point $(-2, 2)$, its eigenvalues are $\lambda = 3 \pm \sqrt{33}$ which means in the nonlinear system, it is also an unstable saddle point.

    For the system with critical point $(4, 4)$, its eigenvalues are $\lambda = -3 \pm i\sqrt{39}$ so in the nonlinear system $(4, 4)$ is an asymptotically stable spiral point.
\end{exercise}


\begin{exercise}{\{5, 6, 7\}abc}
    
    5a. The critical points are $(0, 0)$, $(1, 0)$, $\left(0, \frac{3}{2}\right)$, and $(-1, 2)$. The first three solutions can be found by solving $\frac{dx}{dt} - \frac{dy}{dt} = 0$, and the last solution can be found by solving $x - x^2 = xy = 3y - 2y^2$ as this can be broken up into two equations $1 - x = y$ and $x = 3 - 2y$.

    5b. From Theorem 9.3.2, the linear matrices are of the form $\begin{pmatrix}
        1 - 2x - y & -x \\ -y & 3 - x - 4y
    \end{pmatrix}$ so the matrices for the critical points $(0, 0)$, $(1, 0)$, $\left(0, \frac{3}{2}\right)$, and $\left(-1, 2\right)$ are 

    $$ \begin{pmatrix}
        1 & 0 \\ 0 & 3
    \end{pmatrix}, \begin{pmatrix}
        -1 & -1 \\ 0 & 2
    \end{pmatrix}, \begin{pmatrix}
        -\frac{1}{2} & 0 \\ -\frac{3}{2} & -3 
    \end{pmatrix}, \begin{pmatrix}
        1 & 1 \\ -2 & -4
    \end{pmatrix}$$
    respectively. 
    
    5c. The first matrix has eigenvalues $\lambda = \{1, 3\}$ so $(0, 0)$ is an unstable node in the nonlinear system. \\ 
    The second matrix has eigenvalues $\lambda = \{-1, 2\}$ which means $(1, 0)$ is an unstable saddle point. \\
    The third matrix has eigenvalues $\lambda = \{-\frac{1}{2}, -3\}$ so $\left(0, \frac{3}{2}\right)$ is an asymptotically stable node. \\ 
    The fourth matrix has eigenvalues $\lambda = \frac{1}{2}(3 \pm \sqrt{17})$ so $(-1, 2)$ is an unstable saddle point.


    \vspace{0.3cm}


    6. $(-1, 1) \rightarrow \begin{pmatrix}
        0 & -1 \\ -2 & -2
    \end{pmatrix} \rightarrow \lambda = -1 \pm \sqrt{3}$ so $(-1, 1)$ is an unstable saddle. \\ 
    $(1, 1) \rightarrow \begin{pmatrix}
        0 & -1 \\ 2 & -2 
    \end{pmatrix} \rightarrow \lambda = -1 \pm i$ so $(1, 1)$ is an asymptotically stable spiral point.

    \vspace{0.3cm}

    7. $(0, 0) \rightarrow \begin{pmatrix}
        -2 & 4 \\ 2 & 4
    \end{pmatrix} \rightarrow \lambda = 1 \pm \sqrt{17}$ so $(0, 0)$ is an unstable saddle. 

    $(2, 1) \rightarrow \begin{pmatrix}
        -3 & 6 \\ -4 & 0
    \end{pmatrix} \rightarrow \lambda = \frac{1}{2}(-3 \pm i\sqrt{87})$ so $(2, 1)$ is an asymptotically stable spiral point. 

    $(2, -2) \rightarrow \begin{pmatrix}
        0 & -6 \\ 2 & 0 
    \end{pmatrix} \rightarrow \lambda = \pm i\sqrt{12}$ so $(2, -2)$ is either a center or a spiral point and its stability is indeterminate. 

    $(4, -2) \rightarrow \begin{pmatrix}
        0 & -8 \\ -2 & -4
    \end{pmatrix} \rightarrow \lambda = 2 \pm \sqrt{20}$ so $(4, -2)$ is an unstable saddle point.
\end{exercise}

\begin{exercise}{23}
    
    (Note: (26) is just (25) with $\alpha = -1$.)

    23a. Trivially, $\frac{dx}{dt} = 0 + \alpha(0)(0^2 + 0^2) = \frac{dy}{dt} = 0$ so $(0, 0)$ is a critical point of the system given in (25). To show that $(0, 0)$ is a center, we note that the system can be rewritten as 

    $$\begin{pmatrix}
        x' \\ y'
    \end{pmatrix} = \begin{pmatrix}
        \alpha(x^2 + y^2) & 1 \\ -1 & \alpha(x^2 + y^2)
    \end{pmatrix}\begin{pmatrix}
        x \\ y
    \end{pmatrix} \rightarrow (\lambda - \alpha(x^2 + y^2))^2 + 1 = 0\text{.}$$

    Since $\alpha(x^2 + y^2)$ is 0 at the point $(0, 0)$, $\lambda = \pm i$ which is pure imaginary which means that $(0, 0)$ is a center around the corresponding linear system. 

    \vspace{0.2cm}

    23b. (25) is twice differentiably continuous so (25) is locally linear. 

    23c. $$\frac{dr}{dt} = \frac{xx' + yy'}{r} = \frac{xy + \alpha x^2 r^2 - xy + \alpha y^2 r^2}{r} = \frac{\alpha r^2 (x^2 + y^2)}{r} = \alpha r^3\text{.}$$

    23d. Solving the given differential equation, we have $\alpha t + C = -\frac{1}{2r^2}$ so $\mathlarger{r = \sqrt{\frac{1}{C - 2\alpha t}}}$. If $\alpha < 0$, then as $t \to \infty$, that bottom fraction will go towards infinity so $r$ will go towards 0. 

    23e. If $r(0) = r_0$, then $C = \frac{1}{r_0^2}$ so $\mathlarger{r(t) = \sqrt{\frac{1}{\frac{1}{r_0^2} - 2\alpha t}}}$. Clearly, as the bottom of that denominator becomes 0 (when $t \to 1/2\alpha r_0^2$), $r$ becomes unbounded which translates into $\sqrt{x^2 + y^2}$ becoming unbounded meaning $\frac{dx}{dt}$ and $\frac{dy}{dt}$ will grow astronmically as $t \to 1/2\alpha r_0^2$ when $\alpha > 0$.
\end{exercise}

% 4 - 15 (some/12)
% 23
\end{document}
