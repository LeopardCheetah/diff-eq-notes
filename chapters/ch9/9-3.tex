\documentclass[../../diff_eqs.tex]{subfiles}

\begin{document}

%%%%%%%%%%%%%%%%%%%%%%%%%%%%%%%%%%%%%%%%%%%%%%%%%%%%%%%%%%%%
% docs/syntax:

% definitions
% \begin{definition}[Definition]
%     Definition 1
% \end{definition}

% hr
% \hr

% exercise
% \begin{exercise}{problem number}
%
%    problem starts
% \end{exercise}

%%%%%%%%%%%%%%%%%%%%%%%%%%%%%%%%%%%%%%%%%%%%%%%%%%%%%%%%%%%%

The following table/theorem (theorem 9.3.1 in the book) recaps the stability properties of the origin for the two dimensional linear system $\matr{x'} = \matr{A}\matr{x}$.

\begin{definition}[Stability of $\matr{0}$]
    Let the eigenvalues for $\matr{A}$ be $r_1$ and $r_2$. Then, the critical point $\matr{x} = \matr{0}$ is 
    \begin{enumerate}
        \item asymptotically stable if $r_1$ and $r_2$ have negative real part,
        \item stable if $r_1$ and $r_2$ have $0$ real part (e.g. $r_1$ and $r_2$ are pure imaginary eigenvalues),
        \item unstable if either $r_1$ and $r_2$ have any sort of positive real part.
    \end{enumerate}
\end{definition}

From the table we can conclude that small pertrubations in the roots $r_1$ and $r_2$ only really matter when $r_1$ and $r_2$ are pure imaginary eigenvalues, as any addition of a real part (either positive or negative) will cause the system to spiral inwards towards $\matr{0}$ or outwards to infinity. Thus, while $\matr{0}$ is stable if $r_1$ and $r_2$ have 0 real part, this stability is itself potentially unstable. 

\subsubsection{Linear Approximations}

\begin{definition}[Isolated critical point]
    We say that a critical point $\matr{x}^{\circ}$ is an \textbf{isolated critical point} of the system if there is some disk around $\matr{x}^{\circ}$ with radius $r > 0$ such that there exists no other critical poitns in that disk.     
\end{definition}

Considering the linearization of trajectories around the origin ($\matr{x} = \matr{0})$ for the non-linear system $\matr{x'} = \matr{A}\matr{x} + \matr{g}$, we first assume $\matr{0}$ is an isolated critical point of the system. We also assume that around the critical point (in this case $\matr{x} = \matr{0}$) that $\matr{g}$ is small, or in rigorous terms,

$$\lim_{\matr{x} \to \matr{0}} \frac{|\matr{g}|}{|\matr{x}|} = 0$$

assuming\footnote{It is also possible to convert this limit into polar coordinates ($x = r\cos\theta$, $y = r\sin\theta$, $|\matr{x}| = r$) to make this limit easier to evaluate.} $\matr{g}$ has continuous first partial derivatives. If the above condition is satisfied, the system we described above can then be called a \textbf{locally linear system}. 

\vspace{0.2cm}

Anyways, here's a cool theorem we can use to determine local linearity:

\begin{definition}[Theorem 9.3.2 (p. 410)]
    The system described by 
    $$\matr{x'} = \matr{f}(\matr{x}) = \begin{pmatrix}
        x' \\ y'
    \end{pmatrix} = \begin{pmatrix}
        F(x, y) \\ G(x, y)
    \end{pmatrix}$$
    is locally linear in the neighborhood of a critical point $\matr{x}^{\circ}$ whenever $F$ and $G$ have continuous partial derivatives up to order two (e.g. $F$ and $G$ are twice differentiable).
\end{definition}

If the above condition holds, then the nonlinear system near $\matr{x}^{\circ} = (x^{\circ}, y^{\circ})$ can be approximated by the linear system 

$$\frac{d}{dt}\begin{pmatrix}
    x - x^{\circ} \\ y - y^{\circ}
\end{pmatrix} = \begin{pmatrix}
    F_x(x^{\circ}, y^{\circ}) & F_y(x^{\circ}, y^{\circ}) \\ 
    G_x(x^{\circ}, y^{\circ}) & G_y(x^{\circ}, y^{\circ})
\end{pmatrix}\begin{pmatrix}
    x - x^{\circ} \\ y - y^{\circ}
\end{pmatrix} \ \rightarrow \ \frac{d\matr{u}}{dt} = \frac{d\matr{f}}{d\matr{x}}\left(\matr{x}^{\circ}\right)\matr{u}$$

where $\matr{u} = \begin{pmatrix}
    x - x^{\circ} \\ y - y^{\circ}
\end{pmatrix}$.



The general coefficient matrix in the above equation

$$\matr{J} = \matr{J}[F, G](x, y) = \begin{pmatrix}
    F_x(x, y) & F_y(x, y) \\ 
    G_x(x, y) & G_y(x, y)
\end{pmatrix}$$ 

is called the \textbf{Jacobian}\footnote{Recognize this from Calc 3?} of $F$ and $G$ with repsect to $x$ and $y$. For the linear approximation system above, we need to assume $\det \matr{J}(\matr{x}^{\circ}) \not = 0$ so that $\matr{x}^{\circ}$ is an isolated critical point in our linear approximation system.

Anyways, to relate the properties of stability of linear and locally linear systems, here's a theorem and a table from the textbook:

\begin{center}
    \includegraphics[scale=0.6]{9-3_table.png}
\end{center}

or in summary, except for two special cases ($r_1 = r_2$, $\lambda = 0$), the non-linear terms of the nonlinear system do not affect the stability of the system determined by the linear systems.

If every trajectory approaches the critical point at the origin, then the critical point $\matr{0}$ is said to be \textbf{globally asymptotically stable}.

\vspace{0.2cm}

/// The textbook then goes into more detail about the stability of a damped pendulum around some critical points (Page 413-415). ///



\end{document}
