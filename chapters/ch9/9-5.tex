\documentclass[../../diff_eqs.tex]{subfiles}

\begin{document}

%%%%%%%%%%%%%%%%%%%%%%%%%%%%%%%%%%%%%%%%%%%%%%%%%%%%%%%%%%%%
% docs/syntax:

% definitions
% \begin{definition}[Definition]
%     Definition 1
% \end{definition}

% hr
% \hr

% exercise
% \begin{exercise}{problem number}
%
%    problem starts
% \end{exercise}

%%%%%%%%%%%%%%%%%%%%%%%%%%%%%%%%%%%%%%%%%%%%%%%%%%%%%%%%%%%%


In 9.4, we modelled populations $x$ and $y$ asfighting over some common natural resource (such as food). Here, we will examine situations where one population preys on the other (e.g. foxes and rabbits). Note that $x$ will always represent the population of the species being preyed on, and $y$ will always represent the population of the predatorial species. 

In general, under some assumptions\footnote{See page 428.}, we set up the equations 

$$\frac{dx}{dt} = x(a - \alpha y),$$
$$\frac{dx}{dt} = y(-c - \gamma x)$$

where $a, c, \alpha, \gamma > 0$.\footnote{These equations are also known as the \textbf{Lotka - Volterra equations}.} The physical interpretation of these constants is that $a$ and $c$ are the growth rate of the prey and the death rate of the predator, and $\alpha$ and $\gamma$ are a measure of the effect of the interaction between the species.

\vspace{0.25cm}

A treatise examining solutions of the equations above are given in detail in the textbook, but the main takeaways are that:

\begin{itemize}
    \item The origin ($(0, 0)$) will always be a saddle point.
    \item The ``center'' of the system, $\left(\frac{c}{\gamma}, \frac{a}{\alpha}\right)$, is a critical point upon which all predator-prey trajectories oscillate/circle around.
    \item In particular, $x$ and $y$ can be solved for:
    $$x = \frac{c}{\gamma}(1 + K\cos(\sqrt{ac}t + \phi)), \ y = \frac{a}{\alpha}\left(1 + \sqrt{\frac{c}{a}}K\sin(\sqrt{ac}t + \phi)\right)$$
    for constants $K$, $\phi$. 
\end{itemize}

\end{document}
