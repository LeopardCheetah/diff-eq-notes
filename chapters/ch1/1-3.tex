\documentclass[../../diff_eqs.tex]{subfiles}
\begin{document}

% docs/syntax:

% definitions
% \begin{definition}[Definition]
%     Definition 1
% \end{definition}

% hr
% \hr

% exercise
% \begin{exercise}{problem number}
%
%    problem starts
% \end{exercise}
%%%%%%%%%%%%%%%%%%%%%%%%


\begin{definition}[Ordinary Differential Equation]

    An Ordinary Diffy Q (ODE) is an equation where the unknown function depends on a single independent variable.

    E.g. (LRC Circuit)
    $$L^2 \frac{d^2Q(t)}{dt^2} + R\frac{dQ(t)}{dt} + \frac{1}{C}Q(t) = E(t)$$ 
\end{definition}

\begin{definition}[Partial Differential Equation]

    A Partial Differential Eq (PDE) is when the unknown function depends on several independent variables. 

    E.g. (Wave Equation) 
    $$a^2 \frac{\partial^2 u(x, t)}{\partial x^2} = \frac{\partial^2 u(x, t)}{\partial t^2}$$ 
\end{definition}

(17) - If you have $n$ unknown functions in a system of differential equations, then you gotta have at least $n$ diffy qs to solve that system completely.

\begin{definition}[Order]
    The \textbf{order} of a differential equation is the highest derivative that appears in the differential equation. Thus you can have a \textit{first-order} or \textit{second-order} or \textit{seventh-order} diffy q.

    E.g.: $\mathlarger{ \alpha\frac{d^3x}{dk^3} + \beta\frac{d^2x}{dk^2} + \frac{\alpha}{\beta}x = \gamma }$ is a third-order (ordinary differential) equation (when $\alpha$, $\beta$, and $\gamma$ are constants and $x$ is a function of $k$). 
\end{definition}

Generally then, a differential equation of order $n$ can be represented by the generic $F(t, x(t), x'(t), \dots, x^{(n)}t) = 0$ for some function $x(t)$. Replacing $y = x(t)$, we get that a general $n$th order differential equation is of the form $$F(t, y, y', \dots, y^{(n)}) = 0\text{.}\footnote{(18) Note: We assume it is always possible to solve for the highest derivative -- e.g. we can rearrange to get to the form of $y^{(n)} = f(t, y, y', y'', \dots, y^{(n-1)})$.}$$

\begin{definition}[Linearity]
    A differential equation is said to be \textbf{linear} if $F(t, y, y', \dots, y^{(n)}) = 0$ is a linear function of $t$, $y$, $y'$, \dots, $y^{(n)}$. 

    As such, the general linear diffy q is of the form $0 = c(t) + a_0(t)y + a_1(t)y' + a_2(t)y'' + \dots + a_n(t)y^{(n)}$.
\end{definition}

\begin{definition}[Linearization]
    Linearization is the process of approximating a non-linear diffy q by a linear one. Example given in the textbook is of approximating the motion of an oscillating pendulum.
\end{definition}

(19-20) - Questions of solvability and uniqueness for general differential equations. 


\begin{exercise}{1-4}

    1. Order is 2, and the differential equation is linear. \\
    2. Order is 2, and the differential equation is NOT linear (because of the term $(1 + y^2)\frac{d^2y}{dt^2}$). \\ 
    3. Order is 4, and the differential equation is linear. \\
    4. Order is 2, and the differential equation is non-linear.
\end{exercise}

\begin{exercise}{10}

    (I'm only doing this one because it looks fun)

    We shall verify that $\mathlarger{y = e^{t^2}\left(1 + \int_0^t e^{-s^2} \ ds\right)}$ is a solution to the differential equation $\mathlarger{y' - 2ty = 1}$. 

    First, we substitute $y$ into our equation. 
    $$ \left[e^{t^2} + e^{t^2}\int_0^t e^{-s^2} \ ds\right]' = 1 + 2t \cdot e^{t^2}\left(1 + \int_0^t e^{-s^2} \ ds\right)$$

    Next, we differentiate that left side and simplify the right. 
    $$\longrightarrow 2te^{t^2} + \left(2te^{t^2}\right)\left(\int_0^t e^{-s^2} \ ds \right) + \left(e^{t^2}\right)\left(e^{-t^2}\right) = 1 + 2te^{t^2} + 2te^{t^2}\left(\int_0^t e^{-k^2} \ dk\right) $$

    Finally, we cancel terms and arrive at the equation $$e^{t^2} \cdot e^{-t^2} = 1\text{,}$$

    which is trivially true. Thus, we are done. 
\end{exercise}

\begin{exercise}{11-13}

    Since $y = e^{rt}$, $y^{(n)} = r^ne^{rt}$. Thus, in each of problems 11-13, we're basically just solving a polynomial. To illustrate, consider problem 12:

    $$y'' + y' - 6y = 0 \Longrightarrow r^2e^{rt} + re^{rt} - 6e^{rt} = e^{rt}\left(r^2 + r - 6\right) = 0\text{,}$$

    which is (almost) isomorphic to solving the system $x^2 + x - 6 = 0$. Thus, we yield the solutions $r = 2, 3$ and maybe even $r = -\infty$ (which would make $e^{rt}$ be $0$). \\ 

    Similar solutions follow for 11 and 13.
\end{exercise}

\begin{exercise}{16-18}

    16: 2nd order linear partial differential eq. \\ 
    17: 4th order linear PDE. \\
    18: 2nd order non-linear PDE.
\end{exercise}



\end{document}
