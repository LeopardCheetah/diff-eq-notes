\documentclass[../../diff_eqs.tex]{subfiles}

\begin{document}

% docs/syntax:

% definitions
% \begin{definition}[Definition]
%     Definition 1
% \end{definition}

% hr
% \hr

% exercise
% \begin{exercise}{problem number}
%
%    problem starts
% \end{exercise}
%%%%%%%%%%%%%%%%%%%%%%%%%%%%%%%%%%%%%%%

% textbook page 39, pdf page 53 (i think)

To make a good mathematical model:
\begin{enumerate}
    \item Construct the Model (in accordance with reality).
    \item Analyze the model. Simplify functions/variables if needed, but always make sure the simplifications to the model make physical sense.
    \item Compare the model with outside observations/experiments/data. Also take a look at long run behavior.
\end{enumerate}

// Various examples of mathematical models are shown at this part in the textbook. //

\begin{exercise}{1}
    
    The concentration of dye in the tank can be represented by $\mathlarger{\frac{dA}{dt}}$ where $A(t)$ is the concentration of dye in the tank as a function of time ($t$) in minutes. Specifically, 
    $$\frac{dA}{dt} = \text{rate in} - \text{rate out} = 0 - \frac{A}{200} \cdot 2$$

    which is empirically explained as follows: \\ 
    Since the water coming in is clean, rinse water, there is no dye in that water so the rate at which the dye is coming in the tank is 0. On the flip side, the rate at which the dye is going out is equal to the concentration of the dye ($\frac{\text{Amount}}{\text{Volume}} = \frac{A}{200L}$) times the amount of the water/dye mixture that flows out ($\frac{2L}{\text{min}}$). 

    \vspace{0.3cm}

    Solving for $A(t)$, we find that 
    $$A(t) = e^{-\frac{t}{100}} \ \ (C = 0)\text{.}$$

    As such, to find when the concentration of the dye reaches 1\% of its original value, we simply set $A(t) = 0.01$ and solve, yielding $\boxed{t \approxeq 460.517} (\text{minutes})$ (thanks Wolfram Alpha). 
\end{exercise}

\begin{exercise}{4}
    
    4(a): More clearly, Torricelli's principle states that the outflow velocity $v$ at the outlet is equal to the velocity of a particle in freefall from the height $h$ \textbf{when it reaches the height of the outlet}. \\
    From mechanics, the velocity of a particle in free fall is $v(t) = gt$ since $v_0 = 0$ in our case. Since we don't have $t$, we must resort to finding it by relating the height fallen ($h$) and time: $h = \frac{1}{2}gt^2$ so $t = \frac{2h}{g}$. As such, combining this equation with the previous, we find that $\boxed{v = \sqrt{2gh}}$ as desired. 

    \phantom \\

    4(b): $$\frac{dh}{dt} = \text{How much the height of the water drops with respect to (w.r.t) time} $$
    $$= \frac{\text{How much water goes out w.r.t time}}{\text{Area of the cross section of the tank at height h}} = \frac{-\alpha a v}{A(h)}\text{,}$$

    with that last ``jump'' in reasoning deriving from the fact that how much water goes out of the tank is equal to the velocity of the water ($v$) times the area of the outflow hole ($a$). And apparently since water flow isn't ideal, there's a constant ($\alpha$) tacked on to the ensure the equation matches with real world phenomena.

    \phantom \\ 

    4(c): In this specific scenario, $A(h) = \pi r^2$, $r = 1$, $a = \pi (0.1)^2$, and $h(0) = 3$. Plugging in these values, we get that $\mathlarger{\pi \frac{dh}{dt} = -\alpha \frac{\pi}{100}\sqrt{2gh}}$ so $\mathlarger{\frac{1}{\sqrt{h}}}\frac{dh}{dt} = -\alpha \frac{\sqrt{2g}}{100}$. Integrating and plugging in our initial condition from here, we find that $h(t) = (\sqrt(3) - \frac{\alpha\sqrt{2g}}{200}t)^2$ which means that when $h(t) = 0$, $\boxed{t = \frac{200\sqrt{3}}{\alpha\sqrt{2g}} \approx 130.41}$ (seconds).
\end{exercise}

\begin{exercise}{9}
    
    9(a): We are given that $\mathlarger{\frac{dQ}{dt} = -rQ}$ (so $Q(t) = Q_0e^{-rt}$) and that the halflife of carbon-14 is 5730 years. In other words, $Q(5730) = 0.5Q_0$ (half of the carbon-14 has decayed). As such, $r$ can be solved for and is approximately $\boxed{0.000120968}$. 

    \vspace{0.3cm}

    9(b): Done. See above when I did it. 
    
    \vspace{0.3cm}

    9(c): In this scenario, $\frac{Q(t)}{Q_0} = 0.2$ so $e^{-rt} = 0.2$. With the $r$ we solved for above, $t$ can be solved for and it turns out $\boxed{t = 13304.7}$ (years).
\end{exercise}

\begin{exercise}{14ab}

    14(a): If you're having trouble with this problem, that's ok, I did too. Essentially, use an integrating factor of $e^{kt}$ and then work out the complicated mess that follows. Below is the abridged version of my work. 

    \vspace{0.3cm}

    First, simplify and rearrange to the general form of a differential equation.

    $$\frac{du}{dt} = -k(u - T(t)) \Longrightarrow \frac{du}{dt} + ku = kT(t)$$

    This expression strongly suggests we use an integrating factor of $e^{kt}$, so let's do that!

    $$\rightarrow \int \frac{d}{dt} \left[e^{kt} u(t)\right] = \int ke^{kt}T(t) \ dt$$

    This integral would be terrible if $T(t)$ was weird but thankfully it's not; essentially, we can substitute the expression for $T(t)$ in and simplify the mess from there. 

    $$\rightarrow e^{kt}u(t) = k\int e^{kt} T_0 \ dt + kT_1 \int e^{kt} \cos (\omega t) \ dt$$
    $$= e^{kt}T_0 + kT_1 \cdot R + C$$

    with $R$ being that second integral. Let's now go simplify it! (For those who are following along at home, integrate by parts twice.)

    Using integration by parts ($\int \boxdot \ d \bigstar = \boxdot \bigstar - \int \bigstar \ d \boxdot$) with $\boxdot = \cos(\omega t)$ and $\bigstar = e^{kt}$, our integral ($R$) can now be rewritten as

    $$R = \frac{1}{k}e^{kt}\cos(\omega t) + \frac{\omega}{k} \int e^{kt} \sin(\omega t) \ dt \text{.}$$

    From here, we do integration by parts again ($\boxdot = \sin(\omega t)$ and $\bigstar = e^{kt}$) and we get that 

    $$R = \frac{1}{k}e^{kt}\cos(\omega t) + \frac{\omega}{k} \cdot \frac{1}{k}e^{kt} \sin(\omega t) - \frac{\omega^2}{k^2} R$$ 

    which can thus be simplified to obtain our final result, $$R = e^{kt}\frac{\omega \sin(\omega t) + k \cos (\omega t)}{k^2 + \omega^2}\text{.}$$

    Returning to our original integral and problem (I told you this was messy), we thus reach our final answer for $u(t)$; $$\boxed{u(t) = \frac{C}{e^{kt}} + T_0 + kT_1\left(\frac{\omega \sin(\omega t) + k \cos (\omega t)}{k^2 + \omega^2}\right)}\text{.}$$  

    \hr 

    14(b): [Graphs not pictured] $\tau \approx 3.508$ and $R \approx 9.106$. It's interesting to note that the crossing points for $S(t)$ and $T(t)$ seem to happen exactly at the min/max points of $S(t)$ (which makes sense in the physical interpretation). 
\end{exercise}

\begin{exercise}{14c}

    14(c): Setting the two sides equal to each other, we have that $$R\cos(\omega t - \omega \tau) = \frac{kT_1}{\omega^2 + k^2}(k\cos(\omega t) + \omega \sin (\omega  t))\text{.}$$
    Expanding that left part (and ignoring the fraction for now), we have 
    $$R'(\cos(\omega t)\cos(\omega \tau) + \sin(\omega t)\sin(\omega \tau)) = k\cos(\omega t) + \omega \sin (\omega  t)$$

    where $R'$ is $R$ up to a constant. Anyways, from here, we intuit that $\cos(\omega \tau)$ and $\sin(\omega \tau)$ should be constants that maintain the relative proportions of $\cos (\omega t)$ and $\sin (\omega t)$. As such, we write $\mathlarger{\frac{w}{k} = \frac{\sin(\omega \tau)}{\cos(\omega \tau)}}$ and as such find that $\mathlarger{\tau = \frac{1}{\omega}\arcsin\left(\frac{\omega}{\sqrt{\omega^2 + k^2}}\right)}$. From here, some direct simplification and term comparison leads us to figure out that $R'= \sqrt{\omega^2 + k^2}$. 

    Substituting this back into the original equation, we have 
    $$R\cos(\omega t - \omega \tau) = \frac{kT_1}{\omega^2 + k^2}R'\cos(\omega t - \omega \tau)$$
    
    so $\boxed{\mathlarger{R = \frac{kT_1}{\sqrt{\omega^2 + k^2}}}}$.  
\end{exercise}

%%% pick up later somewhere.


\end{document}
