\documentclass[../../diff_eqs.tex]{subfiles}

\begin{document}

%%%%%%%%%%%%%%%%%%%%%%%%%%%%%%%%%%%%%%%%%%%%%%%%%%%%%%%%%%%%
% docs/syntax:

% definitions
% \begin{definition}[Definition]
%     Definition 1
% \end{definition}

% hr
% \hr

% exercise
% \begin{exercise}{problem number}
%
%    problem starts
% \end{exercise}

%%%%%%%%%%%%%%%%%%%%%%%%%%%%%%%%%%%%%%%%%%%%%%%%%%%%%%%%%%%%

This section is dedicated to solving the end of chapter problems given on page 100 in the textbook. I'm tired of word problems so I'll just be solving the 24 miscallaneous differential equations given. 

\begin{exercise}{1}
    
    I trolled on this problem for a long time :/. \\
    Rewriting our differential equation, we have $\mathlarger{\frac{2}{x}y + \frac{dy}{dx} = x^2}$. Multiplying by a generic integrating factor $\mu$, our equation becomes $\mathlarger{\frac{2}{x}\mu y + \mu\frac{dy}{dx} = x^2\mu}$. Given that we want the left hand side of our equation to look something like $\mathlarger{\frac{d}{dx}[\mu y] = \mu' y + \frac{dy}{dx}\mu}$, we immediately have $\mathlarger{\mu' = \frac{2}{x}\mu \rightarrow \mu(x) = x^2}$. 
    
    As such, we can plug that back into our equation and reformat our differential equation as 
    $$2xy + \frac{dy}{dx}x^2 = x^4 \rightarrow \frac{d}{dx}[x^2y] = x^4$$ 
    and integrate both sides to get 
    $\frac{x^5}{5} + C = x^2y \rightarrow \mathlarger{\boxed{y(x) = \frac{x^3}{5} + \frac{C}{x^2}}}$.
\end{exercise}

\begin{exercise}{2}

    $$\frac{dy}{dx} = \frac{1 + \cos x}{2 - \sin y} \rightarrow 1 + \cos x \ dx = 2 - \sin y \ dy \rightarrow \boxed{x + \sin x + C = 2y + \cos y}\text{.}$$
\end{exercise}

\begin{exercise}{3} 

    We rewrite our diffy q as $\mathlarger{ (-2x - y) + (3 + 3y^2 - x)\frac{dy}{dx} = 0}$. Since $\mathlarger{\frac{\partial M}{\partial y} = \frac{\partial N}{\partial x} = -1}$, our differential equation is \textbf{exact} and we can simply integrate $M = -2x - y$ with respect to $y$ and solve for the constant function to get $\psi(x, y) = -x^2 - xy + 3y + y^3$. With our initial condition $(0, 0)$, our differential equation is implicitly solved by $-x^2 - xy + 3y + y^3 = 0$.
\end{exercise}

\begin{exercise}{4}

    $$\frac{dy}{dx} = 3 - 6x + y - 2xy \longrightarrow \frac{dy}{dx} = (3 + y)(1 - 2x) \Rightarrow y = Ce^{x - x^2} - 3\text{.}$$
\end{exercise}

\begin{exercise}{5}

    $\rightarrow (2xy + y^2 + 1) + (x^2 + 2xy)\frac{dy}{dx}$. $\mathlarger{\frac{\partial M}{\partial y} = \frac{\partial N}{\partial x} = 2x + 2y}$ so our differential equation is exact. As such, $\mathlarger{\psi(x, y) = \int x^2 + 2xy \ dy = \int 2xy + y^2 + 1 \ dx = x^2y + xy^2 + x \ ( = C)}$.
\end{exercise}

\begin{exercise}{6}

    Since there's a $-y$ on the right hand side, we have to rearrange the left hand side to account for it:
    $$x \frac{dy}{dx} + xy = 1 - y \rightarrow \frac{dy}{dx} + \frac{x + 1}{x}y = \frac{1}{x}$$

    As such, we can recognize that an integrating factor of $\mu =  xe^x$ is needed $\left(\mu' = \frac{x + 1}{x}\mu\right)$ and as such we can integrate and get $e^x + C = xe^xy$. With initial condition, our solution thus becomes $\mathlarger{\boxed{y = \frac{1}{x} - \frac{e}{xe^x}}}$.
\end{exercise}

\begin{exercise}{7}

    $$x \frac{dy}{dx} + 2y = \frac{\sin x}{x} \rightarrow x^2\frac{dy}{dx} + 2xy = \sin x \rightarrow \frac{d}{dx}[x^2y] = \sin x \rightarrow y = \frac{C - \cos x}{x^2} \Longrightarrow \boxed{y = \frac{4 + \cos 2 - \cos x}{x^2}}\text{.}$$
\end{exercise}


\begin{exercise}{8}

    This differential equation is exact. 
    $$(2xy + 1) + (x^2 + 2y)y' = 0 \rightarrow \psi(x, y) = x^2y + x + y^2\text{.}$$
\end{exercise}

\begin{exercise}{9}

    Although this differential equation looks like it can be exact, it's actually separable.

    $$(x^2y + xy - y) + (x^2y - 2x^2)\frac{dy}{dx} = 0 \rightarrow (y)(x^2 + x - 1) + (x^2)(y - 2)\frac{dy}{dx} = 0$$ 
    $$\rightarrow -\frac{y(x^2 + x - 1)}{x^2(y - 2)} = \frac{dy}{dx} \rightarrow \frac{y - 2}{y} \ dy = \frac{1 - x - x^2}{x^2} \ dx\text{.}$$

    As such (break apart the fractions!), $\mathlarger{\boxed{y - 2 \ln y = -x - \ln x - \frac{1}{x} + C}}$.
\end{exercise}


\begin{exercise}{10}

    This time, the differential equation is exact. This makes our work super easy as we can just recognize $\psi(x, y) = \frac{x^3}{3} + xy + e^y$ which captures all solutions when paired with a level curve.
\end{exercise}

\begin{exercise}{11}

    This differential equation is also exact. $\psi(x, y) = \frac{x^2}{2} + xy + y^2$ so with point $(2, 3)$, the solution is $\mathlarger{\frac{x^2}{2} + xy + y^2 = 17}$.
\end{exercise}

\begin{exercise}{12}
    We can rewrite our differential equation to be separable and seperate to get 
    $$\rightarrow \ln y = C + \int \frac{1 - e^x}{1 + e^x}\text{.}$$

    Since this kind of looks $u$-subbable, we let $u = e^x + 1$; correspondingly, $du = e^x \ dx$ and $1 - e^x = 2 - u$. As such, our integral then becomes

    $$= C + \int \frac{2 - u}{u(u - 1)} \ du = C + \int \frac{1}{u - 1} - \frac{2}{u} \ du\text{.}$$

    As such, now integratable, we integrate and simplify to get $\mathlarger{\boxed{y = Ce^x \cdot (e^x + 1)^{-2}}}$.
\end{exercise}

\begin{exercise}{13}

    This big scary equation is actually an exact differential equation. Solving, we get $\psi(x, y) = e^{-x}\cos y + e^{2y}\sin x$.
\end{exercise}

\begin{exercise}{14}
    
    $$-3y + \frac{dy}{dx} = e^{2x} \rightarrow e^{-x} = -3e^{-3x}y + e^{-3x}\frac{dy}{dx} \rightarrow -e^{-x} + C = e^{-3x}y\text{.}$$
\end{exercise}

\begin{exercise}{15}
    Using integrating factor $e^{2x}$, we get $$e^{2x}y = \int e^{-x^2} \ dx + C$$
    so $\mathlarger{y = e^{-2x} \frac{\sqrt{\pi}}{2} \text{erf}(x) + 3e^{-2x}}$.
\end{exercise}

\begin{exercise}{16}

    Again, another big scary fraction that turns super tame when we rearrange it and find it's an exact differential equation. $\psi(x, y) = xy^3 + 2xy - x^3$ is the parent solution.
\end{exercise}

\begin{exercise}{17}
    
    $$y' = e^{x + y} \rightarrow e^{-y} \ dy = e^x \ dx \rightarrow -y = \ln(-e^x) \rightarrow y = -x(1 + i\pi)\text{.}$$
\end{exercise}

\begin{exercise}{18}

    $\mathlarger{\psi(x, y) = 2xy^2 + 3x^2y - 4x + y^3}$.
\end{exercise}


\begin{exercise}{19}

    Note: This problem is functionally equivalent to problem 6. The integrating factor is the same ($xe^x$) so the problem is basically identical.

    $$t \frac{dy}{dt} + (t + 1)y = e^{2t} \rightarrow e^{3t} = te^t\frac{dy}{dt} + (t + 1)e^t y \rightarrow \frac{e^{3t}}{3} + C = te^ty \rightarrow y = \frac{e^{2t}}{3t} + \frac{C}{te^t}\text{.}$$
\end{exercise}

\begin{exercise}{20}
    
    This problem is very tricky and I gave up on the problem (although I was close in trying to make a substitution). Essentially, to clear things up, we set $y = nx$ and correspondingly find $\mathlarger{\frac{dy}{dx} = n + \frac{dn}{dx}x}$. We then substitute this new $\frac{dy}{dx}$ back into our differential equation and solve. 

    Eventually, we find $-e^{-n} = \ln x + C$ so $e^{-y/x} + \ln x = C$ is our solution.
\end{exercise}

\begin{exercise}{21}

    This problem is weird. Also, I wouldn't take the hint that the textbook gives. 
    
    \vspace{0.3cm}

    Motivated by how bad the differential equation is when we first write it out $\mathlarger{\left(\frac{dy}{dx} = \frac{x}{y}\cdot \frac{1}{x^2 + y^2}\right)}$, we are somewhat motivated in making the substitution $v = x^2 + y^2$ to clean things up. Indeed, we can rewrite the whole differential equation in terms of $v$ and $x$: 

    $$v = x^2 + y^2 \rightarrow \frac{dv}{dx} = 2x + 2y\frac{dy}{dx} \rightarrow \frac{dy}{dx} = \frac{\frac{dv}{dx} - 2x}{2y}$$
    $$\Longrightarrow \frac{\frac{dv}{dx} - 2x}{2y} = \frac{x}{y}\cdot\frac{1}{v} \Rightarrow \frac{dv}{dx} = 2x\left(\frac{v+1}{v}\right)$$

    from which the equation is separable and then easily integratable. 

    \hr 

    Lesson of the day then is to use clever substitutions to turn a non-linear differential equation (with term $y^3\frac{dy}{dx}$) into a linear, solvable differential equation.
\end{exercise}


\begin{exercise}{22}

    Use the substitution $y = nx$ $\mathlarger{\left(\frac{dy}{dx} = n + \frac{dn}{dx}\right)}$ and rearrange:

    $$\rightarrow n + \frac{dn}{dx}x = \frac{(n + 1)x}{(1 - n)x} \rightarrow \frac{dn}{dx} = \frac{1 + n^2}{1 - n} \cdot \frac{1}{x}$$

    which is clearly solvable and might be integratable and might result in a solution that isn't deranged and unwritable.
\end{exercise}


\begin{exercise}{23}

    I'm not even sure how this still works but once again the substitution $y = nx$ works (This time I figured it out by myself!!).

    $y = nx$ so $\mathlarger{\frac{dy}{dx} = n + x\frac{dn}{dx}}$. Replacing this in our differential equation, we get:

    $$3n^2x^2 + 2nx^2 - \left(n + x \frac{dn}{dx}\right)(2nx^2 + x^2) = 0 \rightarrow 3n^2 + 2n - \left[2n^2 + n + (2n + 1)x\frac{dn}{dx}\right] = 0$$

    $$\rightarrow n^2 + n = (2n + 1)x\frac{dn}{dx} \rightarrow \frac{1}{x} \ dx = \frac{1}{n} + \frac{1}{n+1} \ dn$$ 

    from which the equation is easily solvable :).

    Lesson learned: Always substitute $y = nx$ if things look funky.
\end{exercise}


\begin{exercise}{24}

    Not even going to attempt this one. I've wasted too much time on this stupid problem.

    The solution is to divide the equation by $y^2$ and recognize that you can find an integrating factor $\mu$ for the exact differential equation by trying to find $\frac{d\mu}{dx}$. Good luck o7.
\end{exercise}

\end{document}
