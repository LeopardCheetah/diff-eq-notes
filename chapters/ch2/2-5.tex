\documentclass[../../diff_eqs.tex]{subfiles}

\begin{document}

% docs/syntax:

% definitions
% \begin{definition}[Definition]
%     Definition 1
% \end{definition}

% hr
% \hr

% exercise
% \begin{exercise}{problem number}
%
%    problem starts
% \end{exercise}
%%%%%%%%%%%%%%%%%%%%%%%%%%%%%%%%%%%%%%%

This section mostly examines the logistic equation modelling population growth in the context of differential equations and some goes somewhat in-depth about it.

\begin{definition}[Autonomous Differential Equation]

    A differential equation is \textbf{autonomous} if the independent variable does not appear explicitly. They have the form 

    $$\frac{dy}{dt} = f(y)\text{.}$$

    Note that the zeros of $f(y)$ ({$z | f(z) = 0$}) are called \textbf{critical points}.
\end{definition}

\begin{definition}[Logistic (Verhulst) Equation]
    The equation of the form 
    $$\frac{dy}{dt} = r\left(1 - \frac{y}{K}\right)y$$
    which is commonly used to represent population growth.

    Notably, $K$ is the \textbf{carrying capacity} for the population and $r$ is called the \textbf{intrinsic growth rate}.
\end{definition}

(65) - Logistic Growth with a Threshold: \\
The equation 
$$\frac{dy}{dt} = -r\left(1 - \frac{y}{T}\right)\left(1 - \frac{y}{K}\right)y$$
represents a logistic growth function with a threshold (where $0 < T < K$) \textemdash graphing the solution to this function reveals that all initial values $y_0 > T$ gravitate towards $K$ like in logistic growth yet all initial values $y_0 < T$ eventually fade to 0.


\begin{exercise}{5}
    Semistable Equilibrium Solutions

    5a: Trivial since if $y \not = 1$, then $(1 - y)^2$ will be non-zero by the trivial inequality.

    \vspace{0.3cm}

    5c: $$\frac{dy}{dt} = k(1 - y)^2 \rightarrow \frac{1}{(1 - y)^2} \ dy = k \ dt \rightarrow \frac{1}{1 - y} = kt + C\text{.}$$

    With initial condition $y(0) = y_0$, $C = \frac{1}{1 - y_0}$ so $y(t) = \mathlarger{\boxed{1 - \frac{1 - y_0}{kt - kty_0 + 1}}}$.
\end{exercise}

\begin{exercise}{17a}

    17(a): Separating, we get $$\frac{1}{\ln K - \ln y} \frac{1}{y} \ dy = r \ dt\text{.}$$
    Integrating, we have $$rt + C = \int \frac{1}{y} \ dy \ \frac{1}{\ln K - \ln y}$$
    from which we can make the substitution $\ln y = u$ (and $\frac{1}{y} \ dy = du$) so the RHS becomes $\mathlarger{\int \frac{1}{\ln k - n} \ dn}$. From here, integrating gets us $$rt + C = -\ln\left|\ln\left(\frac{K}{y}\right)\right|$$ and a multitude of substitutions leads us to find $C = \ln \frac{y_0}{K}$ so 


    $$y(t) = K \text{exp}\left(\ln\left(\frac{y_0}{K}\right)e^{-rt}\right)\text{.}$$
\end{exercise}

% page 68
\begin{exercise}{18? 19? 20?}


\end{exercise}

\end{document}
