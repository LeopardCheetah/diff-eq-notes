\documentclass[../../diff_eqs.tex]{subfiles}

\begin{document}

% docs/syntax:

% definitions
% \begin{definition}[Definition]
%     Definition 1
% \end{definition}

% hr
% \hr

% exercise
% \begin{exercise}{problem number}
%
%    problem starts
% \end{exercise}
%%%%%%%%%%%%%%%%%%%%%%%%%%%%%%%%%%%%%%%

This section mostly examines the logistic equation modelling population growth in the context of differential equations and some goes somewhat in-depth about it.

\begin{definition}[Autonomous Differential Equation]

    A differential equation is \textbf{autonomous} if the independent variable does not appear explicitly. They have the form 

    $$\frac{dy}{dt} = f(y)\text{.}$$

    Note that the zeros of $f(y)$ $\{z \ | \ f(z) = 0\}$ are called \textbf{critical points}.
\end{definition}

(Note: There's some definitions of \textbf{asymptotically stable} points and other terms that aren't noted down here (I don't care about those terms).)

\begin{definition}[Logistic (Verhulst) Equation]
    The equation of the form 
    $$\frac{dy}{dt} = r\left(1 - \frac{y}{K}\right)y$$
    which is commonly used to represent population growth.

    Notably, $K$ is the \textbf{carrying capacity} for the population and $r$ is called the \textbf{intrinsic growth rate}.
\end{definition}

(65) - Logistic Growth with a Threshold: \\
The equation 
$$\frac{dy}{dt} = -r\left(1 - \frac{y}{T}\right)\left(1 - \frac{y}{K}\right)y$$
represents a logistic growth function with a threshold (where $0 < T < K$) \textemdash graphing the solution to this function reveals that all initial values $y_0 > T$ gravitate towards $K$ like in logistic growth yet all initial values $y_0 < T$ eventually fade to 0.


\begin{exercise}{5}
    Semistable Equilibrium Solutions

    5a: Trivial since if $y \not = 1$, then $(1 - y)^2$ will be non-zero by the trivial inequality.

    \vspace{0.3cm}

    5c: $$\frac{dy}{dt} = k(1 - y)^2 \rightarrow \frac{1}{(1 - y)^2} \ dy = k \ dt \rightarrow \frac{1}{1 - y} = kt + C\text{.}$$

    With initial condition $y(0) = y_0$, $C = \frac{1}{1 - y_0}$ so $y(t) = \mathlarger{\boxed{1 - \frac{1 - y_0}{kt - kty_0 + 1}}}$.
\end{exercise}

\begin{exercise}{13}
    
    To find the inflection points of $y(t)$, we can find the min/maxes of $f(y)$ since $y'(t) = f(y)$ (kinda). As such, we find those minimum/maximum points by setting $\mathlarger{\frac{df}{dy} = 0}$ then solving.
    
    Starting from (17), we apply a 3-part product rule to get 
    $$f'(y) = -r\left[\left(1 - \frac{y}{T}\right)\left(-\frac{1}{K}\right)y + \left(-\frac{1}{T}\right)\left(1 - \frac{y}{K}\right)y - \left(1 - \frac{y}{T}\right)\left(1 - \frac{y}{K}\right)\right]\text{.}$$

    The inside function is merely a quadratic (a very messy one) $\mathlarger{\left[y^2 \left(\frac{3}{KT}\right) - y\left(\frac{2}{K} + \frac{2}{T}\right) + 1\right]}$, which can be bashed using the quadratic formula to get the desired solutions.
\end{exercise}

\begin{exercise}{17a}

    17(a): Separating, we get $$\frac{1}{\ln K - \ln y} \frac{1}{y} \ dy = r \ dt\text{.}$$
    Integrating, we have $$rt + C = \int \frac{1}{y} \ dy \ \frac{1}{\ln K - \ln y}$$
    from which we can make the substitution $\ln y = u$ (and $\frac{1}{y} \ dy = du$) so the RHS becomes $\mathlarger{\int \frac{1}{\ln k - n} \ dn}$. From here, integrating gets us $$rt + C = -\ln\left|\ln\left(\frac{K}{y}\right)\right|$$ and a multitude of substitutions leads us to find $C = \ln \frac{y_0}{K}$ so 


    $$y(t) = K \text{exp}\left(\ln\left(\frac{y_0}{K}\right)e^{-rt}\right)\text{.}$$
\end{exercise}

% page 68
\begin{exercise}{19}

    19(a): Assuming $E < r$, to solve we set the RHS equal to 0:
    $$r\left(1 - \frac{Y}{K}\right)y - Ey = 0 \rightarrow y\left(r - \left(1 - \frac{y}{K}\right) - E\right) = 0$$
    meaning either $y_1 = 0$ or the inside function is 0. Simplifying that inside function reduces to the second solution, $y_2 = K\left(1 - \frac{E}{r}\right)$.

    \vspace{0.3cm}
    
    19(b): The inside function is a parabola pointed downwards. As such, function values for $y$ values right below the first solution $y_1 = 0$ are negative and function values for $y$ values right above 0 are positive. This translates to values drifting away from $y = 0$ as negative values become even more negative and positive values continue to get more positive making $y_1$ an unstable equilibrium. \\ 
    On the other hand, values slightly above $y_2$ have a negative $\frac{dy}{dt}$ and values slightly below $y_2$ have a positive $\frac{dy}{dt}$ so overall those values will drift towards $y_2$. Thus, $y_2$ is an asymptotically stable solution.

    \vspace{0.3cm}

    19(c): Since $Y = E \cdot y_2$, $Y = EK\left(1 - \frac{E}{r}\right)$.

    19(d): The maximum of a parabola lies at its vertex $-\frac{b}{2a}$. In this case, $Y(E) = -E^2\frac{K}{r} + EK$ is maximized when $E = \frac{r}{2}$ and correspondingly $Y_m = \frac{KR}{4}$.
\end{exercise}

\begin{exercise}{27}

    27(a): The limiting value of $x(t)$ as $t \to \infty$ is $x = \min\{p, \ q\}$. Slightly above this value $\frac{dx}{dt}$ is negative (meaning $x(t)$ decreases) and slightly below this value $\frac{dx}{dt}$ is positive.


    We can solve the differential equation using partial fractions:
    $$\frac{dx}{dt} = \alpha(p - x)(q - x) \rightarrow \int \frac{1}{(p - x)(q - x)} \ dx = \alpha t + C \rightarrow \frac{1}{q - p} \int \frac{1}{p - x} - \frac{1}{q - x} \ dx = \alpha t + C$$

    so $\mathlarger{\frac{1}{q - p} \left( -\ln |p - x| + \ln |q - x|\right) = \alpha t + C}$ 
    with $\mathlarger{C = \frac{\ln(q/p)}{q - p}}$. (The simplification afterwards for an explicit form of $x(t)$ gets really messy.)

    \vspace{0.3cm}

    27(b): As seen in problem 5, this is a equation with a semistable equilibrium solution $x(t) = p$ which happens when $t \to \infty$ with initial condition $x(0) = 0$. A simple integration shows $\int \frac{1}{(p - x)^2} \ dx = \frac{1}{p - x} + C$ and thus $x(t) = p(1 - \frac{1}{p\alpha t + 1})$.
\end{exercise}




\end{document}
