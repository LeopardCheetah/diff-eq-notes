\documentclass[../../diff_eqs.tex]{subfiles}

\begin{document}

% docs/syntax:

% definitions
% \begin{definition}[Definition]
%     Definition 1
% \end{definition}

% hr
% \hr

% exercise
% \begin{exercise}{problem number}
%
%    problem starts
% \end{exercise}
%%%%%%%%%%%%%%%%%%%%%%%%%%%%%%%%%%%%%%%


This long section is just kind of a proof of the existence and uniqueness of solutions of a continuous differential equation.

The proof uses a method called ``method of successive approximations'' which solves a differential equation by considering the solution $\mathlarger{\phi(t) = \lim_{n \to \infty} \phi_n(t)}$ where 
$$\phi_{n + 1}(t) = \int_0^t f(s, \phi_n(s)) \ ds$$

($f$ is the thing you get when you rewrite a differential equation as $\frac{dy}{dt} = f(t, y)$.)


\vspace{0.3cm}


Essentially this whole section is just dedicated to the proof of an elementary yet powerful (but also cumbersome) theorem.


\begin{exercise}{1}

    $\mathlarger{\frac{dy}{dt} = (t + 1)^2 + (y + 2)^2}$.
\end{exercise}

\end{document}
