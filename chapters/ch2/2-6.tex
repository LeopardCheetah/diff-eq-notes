\documentclass[../../diff_eqs.tex]{subfiles}

\begin{document}

% docs/syntax:

% definitions
% \begin{definition}[Definition]
%     Definition 1
% \end{definition}

% hr
% \hr

% exercise
% \begin{exercise}{problem number}
%
%    problem starts
% \end{exercise}
%%%%%%%%%%%%%%%%%%%%%%%%%%%%%%%%%%%%%%%


\begin{definition}[Exact Differential Equation]
    If we can identify a function $\psi(x, y)$ such that $\frac{\partial \psi}{\partial x} = M(x, y)$ and $\frac{\partial \psi}{\partial y} = N(x, y)$ for a given differential equation $M(x, y) + N(x, y)y' = 0$, then that differential equation is \textbf{exact}
    and the solutions are implicitly given by $\psi(x, y) = c$ for arbitrary $c$.
\end{definition}

While this itself is daunting, \href{https://en.wikipedia.org/wiki/Symmetry_of_second_derivatives}{Clairut's Theorem} simplifies this tremendously for us as under the assumption that $M$, $N$, $\frac{\partial M}{\partial y}$, $\frac{\partial N}{\partial x}$ are continuous in some closed region, Clairut's theorem tells us 

$$\frac{\partial}{\partial x}\left(\frac{\partial f}{\partial y}\right) = \frac{\partial}{\partial y}\left(\frac{\partial f}{\partial x}\right)\text{.}$$


We can then substitute our function $\psi$ for the generic function $f$ and simplify:

$$\frac{\partial}{\partial x}\left(\frac{\partial \psi}{\partial y}\right) = \frac{\partial}{\partial y}\left(\frac{\partial \psi}{\partial x}\right) \longrightarrow \frac{\partial}{\partial x}(N) = \frac{\partial}{\partial y}(M)$$


concluding that the function $\psi$ only exists if $\mathlarger{\boxed{\frac{\partial M}{\partial y} = \frac{\partial N}{\partial x}}}$. 

\vspace{0.3cm}

Now with this simple test at our side, if a function $\psi$ exists for an exact differential equation, since $M = \frac{\partial \psi}{\partial x}$, we can simply integrate $M$ with respect to $x$ with the constant term being replaced by an arbitrary function $C(y)$. To then solve for $C$, take the derivative of the integrated thing and compare terms.


\hr 

%%%%%%%% end exact diffy qs. %%%%%%%%%%%

Returning to the concept of integrating factors from Section 2.1, another technique to solving general differential equations is to multiply both sides by a specific integrating factor such that the resulting differential equation is exact.

In symbols, we find a function $\mu$ as an integrating factor
$$M(x, y) + N(x, y)y' = 0 \longrightarrow (\mu(x, y) \cdot M) + (\mu(x, y) \cdot N)y' = 0$$

so $\mathlarger{\frac{\partial (\mu M)}{\partial y} = \frac{\partial (\mu N)}{\partial x}}$ to make this new differential equation an \textbf{exact differential equation}.

\vspace{0.3cm}

Unfortunately, this is super duper difficult (there is no good way to find $\mu$) in general. As such, we have to gimmick a little and hope that $\mu$ is a function of one variable which simplifies things considerably. 

Assuming $\mu = \mu(y)$, we can use the product rule on both sides of the above equation to get 

$$ \mu(y)\frac{\partial M}{\partial y} + M \frac{d \mu(y)}{dy} = \mu(y)\frac{\partial N}{\partial x} + N \frac{\partial \mu(y)}{\partial x} = \mu(y)\frac{\partial N}{\partial x}$$

which can we rearrange and find $\mathlarger{\frac{d \mu}{dy} = \frac{\frac{\partial N}{\partial x} - \frac{\partial M}{\partial y}}{M}\mu}$ from which $\mu$ can be integrated and solved for.\footnote{A similar equation arises if you assume $\mu$ is a function of $x$.} 

\vspace{0.3cm}


\begin{exercise}{1-8}

    1. This differential equation is exact; $\psi(x, y) = x^2 + 3x + y^2 - 2y ( \ = c)$. \\ 
    2. Not exact. \\ 
    3. Exact; $\psi(x, y) = x^3 - x^2 y + 2x + 2y^3 + 3y$. \\
    4. Exact (rearrange to get $(ax + by) + (bx + cy)y' = 0$); $\psi(x, y) = \frac{a}{2}x^2 + bxy + \frac{c}{2}y^2$. \\ 
    5. Not exact. \\
    6. Exact; $\psi(x, y) = e^{xy} \cos(2x) + x^2 - 3y$. \\ 
    7. Exact; $\psi(x, y) = y \ln x + 3x^2 - 2y$. \\ 
    8. Exact; $\psi(x, y) = -\frac{1}{\sqrt{x^2 + y^2}}$ (via $u$-substitution).
\end{exercise}

\begin{exercise}{14}

    A differential equation is exact if $M_y = N_x$. Since $M = M(x)$, $\frac{\partial M(x)}{\partial y} = 0$ (it's basically a constant) and similarly $\frac{\partial N(y)}{\partial x} = 0$ so the separable equation is exact.
\end{exercise}

\begin{exercise}{17}

    From the derivation I did previously, $\mathlarger{\frac{d \mu}{dy} = \frac{\frac{\partial N}{\partial x} - \frac{\partial M}{\partial y}}{M}\mu = \frac{N_x - M_y}{M}\mu}$. Thus, we have $\mu' = Q(y)\mu$ so $\mathlarger{\mu(y) = e^{\int Q(y) \ dy}}$ as given.
\end{exercise}

\begin{exercise}{18-21}

    18. Assuming $\mu = \mu(x)$, we find the integrating factor to be $\mu(x) = e^{3x}$. Integrating, we find $\psi(x, y) = x^2ye^{3x} + \frac{1}{3}y^3e^{3x}$. \\ 
    19. The integrating factor is $e^{-x}$ and $\psi(x, y) = ye^{-x} - e^x - e^{-x}$. \\ 
    20. The integrating factor is $y$ and $\psi(x, y) = xy + y \cos y - \sin y$. \\ 
    21. The integrating factor is $\mu(y) = \frac{e^{2y}}{y}$ and integration reveals $\psi(x, y) = xe^{2y} - \ln |y|$. 
\end{exercise}



\end{document}
