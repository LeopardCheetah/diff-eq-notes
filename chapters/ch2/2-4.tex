\documentclass[../../diff_eqs.tex]{subfiles}

\begin{document}

% docs/syntax:

% definitions
% \begin{definition}[Definition]
%     Definition 1
% \end{definition}

% hr
% \hr

% exercise
% \begin{exercise}{problem number}
%
%    problem starts
% \end{exercise}
%%%%%%%%%%%%%%%%%%%%%%%%%%%%%%%%%%%%%%%


% 2.4, textbook page 51, pdf page 65
%%% todo: switch this to definition style
\begin{definition}[Existence and Uniqueness Theorem (for First-Order Linear Equations)]

    If $p$ and $q$ are continuous functions on an interval $I$: $\alpha < t < \beta$, then there exists a unique function $y = \phi(t)$ that satisfies the differential equation
    $$\frac{dy}{dt} + p(t)y = q(t)$$
    with initial condition $y(\gamma) = \delta$ with $\delta$ being an arbitrary intial value and $\gamma \in I$.

\end{definition}

Note that this theorem asserts both the \textbf{existence} and \textbf{uniqueness} of a solution to a given first order linear differential equation.

\vspace{0.7cm}

\begin{definition}[Existence and Uniqueness Theorem for (for First-Order Nonlinear Equations)]

    Let $f$ and $\frac{\partial f}{\partial y}$ be continuous in some rectangle $\alpha < t < \beta$, $\gamma < y < \delta$ containing the point ($t_0$, $y_0$). Then, in some interval $t_0 - h < t < t_0 + h$ contained in $\alpha < t < \beta$, there is a unique solution $y = \phi(t)$ to the differential equation 

    $$\frac{dy}{dt} = f(t, y), \; y(t_0) = y_0 \text{.}$$
    
\end{definition}

Note: The conclusion that a differential equation exists can be established based on the continuity of $f$ alone. However, the given solution $\phi$ may or may not be unique.

\vspace{0.3cm}

(Example differential equations are then solved through examples to highlight example applications of the existence theorems in practice and show the necessity of the conditions.)

\vspace{0.3cm}

Note: For first order linear differential equations, its possible points of discontinuity/singularity can be found by examining the discontinuity of the coefficients $p$ and $g$.


\vspace{0.3cm}

Note (page 56): General solutions may not exist for non-linear differential equations. That is, there may be functions $\phi$ and $\psi$ that both satisfy the given differential equation yet not be of the same form as each other.


\begin{exercise}{1-4}
    
    1. Rearranging, we get that the differential equation is $$\frac{dy}{dt} + \frac{\ln t}{t - 3}y = \frac{2t}{t - 3}\text{.}$$
    Clearly, from the $(t - 3)$ in the denominator, our end function $y = \phi(t)$ will be discontinuous at $t = 3$. Similarly, since $\ln t$ is discontinuous for all $t \leq 0$\footnote{$\ln(-1) = i\pi$}, $\phi$ may be discontinuous at $t \leq 0$. As such, we know for sure that $\phi$ exists over the intervals $0 < t < 3$ and $3 < t < \infty$. \\
    2. The function $\tan(t)$ is discontinuous at $\mathlarger{\{\dots, -\frac{3\pi}{2}, -\frac{\pi}{2}, \frac{\pi}{2}, \frac{3\pi}{2}, \dots\}}$ so the solution function $y = \phi(t)$ is certain to exist over the intervals $\mathlarger{n \pi + \frac{\pi}{2} < t <  (n + 1)\pi + \frac{\pi}{2}}$ for all integer values of $n$. \\
    3. After rewriting the differential equation, $(4 - t^2)$ appears in the denominator of $p$ and $g$ meaning $\phi$ is certain to exist over the intervals $(\-infty, -2)$, $(-2, 2)$, $(2, \infty)$. \\ 
    4. $\cot(t)$ is discontinuous at every multiple of $\pi$ and $\ln(t)$ (as mentioned before) is discontinuous for all $t \leq 0$. As such, $\phi$ is certain to exist over the intervals $n\pi < t < (n + 1)\pi$ for all nonnegative values of $n$.
\end{exercise}

% todo: refine wording.
\begin{exercise}{9}
    
    We seperate the variables and integrate to get $\mathlarger{\frac{1}{2}y^2 = -2t^2 + C}$. With initial condition $y(0) = y_0$, $\mathlarger{C = \frac{1}{2} y_0^2}$. As such, the general solution to our differential equation is
    $$y = \pm \sqrt{y_0^2 - 4t^2}$$ with the $\pm$ sign indicating directionality of the answer. Notably, if $y_0 < 0$, then the negative sign of the square root is taken while if $y_0 > 0$, the positive sign of the square root is taken. Also, note that $|t|$ must be less than $|y|/2$ to make the equation work; if $|t| = |y|/2$, then $y(t_0) = 0$ for some $t_0$ yet then our original differential equation ($y' = -4t/y$) has no solution for $y'$ when $y = 0$.
\end{exercise}

\begin{exercise}{10}
    
    The solution to the given differential equation is $\mathlarger{y = -\frac{1}{t^2 - \frac{1}{y_0}}}$. Given the discontinuity in the denominator, $t$ cannot be equal to $\mathlarger{\frac{1}{\sqrt{y_0}}}$ and as such the domain of $t$ will be restricted to $\mathlarger{-\frac{1}{\sqrt{y_0}} < t < \frac{1}{\sqrt{y_0}}}$ if $y_0 \geq 0$ (and $t$ is unrestricted if $y_0 < 0$.)
\end{exercise}

\begin{exercise}{11}

    % t + C = 1/2y^2
    The solution to the differential equation ends up being $\mathlarger{t + \frac{1}{2y_0^2} = \frac{1}{2y^2}}$ or $\mathlarger{y = \pm \frac{1}{\sqrt{2(t + 1/2y_0^2)}}}$. Like in exercise 9, the positive branch of the square root is taken if $y_0 > 0$ and the negative branch taken if $y_0 < 0$. Otherwise, the domain of $t$ is restricted to $t > -\frac{1}{2y_0^2}$.
\end{exercise}

\begin{exercise}{18}

    (a) Verified. \\ 
    (b) $\mathlarger{\frac{\partial f}{\partial y} = \frac{\partial}{\partial y} \left(\frac{-t + \sqrt{t^2 + 4y}}{2}\right) = \frac{1}{\sqrt{t^2 + 4y}}}$ is not continuous with the initial condition $y(2) = -1$. As such, while a solution to the differential equation $\phi$ can be guaranteed, the solution $\phi$ may or may not be unique as not all requirements of Theorem 2.4.2 are satisfied. \\ 
    (c) Given that we know the second solution $y_2(t)$, it's pretty easy to check no constant $c$ plugged into the equation $y = ct + c^2$ can yield a $t^2$ term like in $y_2(t)$. Otherwise, fixing the initial condition $y(2) = -1$, plugging $t = 2$ and $y = -1$ \textbf{only} yields the solution $c = -1$.
\end{exercise}


\begin{exercise}{21}
    
    By direct simplification,
    $$\frac{dy}{dt} + p(t)y \longrightarrow \frac{d}{dt}[y_1(t) + y_2(t)] + p(t)(y_1(t) + y_2(t)) = y_1'(t) + p(t)y_1(t) + y_2'(t) + p(t)y_2(t) = g(t)\text{.}$$
\end{exercise}

Discontinuous coefficients problems look daunting but really you're just solving more or less the same differential equation twice then gluing the pieces together.   

\begin{exercise}{26}

    Here, we solve $y' + 2y = g(t)$ by multiplying both sides by the integrating factor of $e^{2t}$. We then evaluate both cases ($t > 1$, $0 \leq t \leq 1$) separately, and glue together both functions when $t = 1$.

    With the case of when $g(t) = 1 \ (0 \leq t \leq 1)$,
    $$y' + 2y = 1 \rightarrow e^{2t}\frac{dy}{dt} + 2e^{2t}y = e^{2t} \longrightarrow \int \frac{d}{dt}\left[e^{2t}y\right] = \int e^{2t} \ dt \Longrightarrow y_\alpha(t) = \frac{e^{2t} + C_\alpha}{2e^{2t}}\text{.}$$

    Since we have the initial condition of $y(0) = 0$, $C_\alpha = -1$ and $\mathlarger{y_\alpha(t) = \frac{e^{2t} - 1}{2e^{2t}}}$. 

    The case when $g(t) = 0$ is simple enough to not warrant covering and the final equation obtained is $\mathlarger{y_\beta(t) = \frac{C_\beta}{e^{2t}} \ (t > 1)}$. To make the overall function $y(t)$ continuous ($y = y_\alpha \cup y_\beta$), we set $y_\alpha(1) = y_\beta(1)$. Namely, $\mathlarger{y_\alpha(1) = \frac{e^2 - 1}{2e^2}}$ and $\mathlarger{y_\beta(1) = \frac{C_\beta}{e^2}}$ so $\mathlarger{C_\beta = \frac{e^2 - 1}{2}}$.

    Overall then,

    $$\mathlarger{y(t) = \begin{cases} \frac{e^{2t} - 1}{2e^{2t}} \ \ 0 \leq t \leq 1 \\ \frac{e^2 - 1}{2e^{2t}} \ \ t > 1\end{cases}\text{.}}$$
\end{exercise}


\begin{exercise}{27}
    
    Setting up the differential equation to be separable, we obtain 
    $$\frac{1}{p(t) y} \ dy = - dt\text{.}$$

    For the case when $p(t) = 2$, $y_\alpha$ is correspondingly 
    $$y_\alpha(t) = e^{-2t}$$ 
    after plugging in initial condition $y(0) = y_\alpha(0) = 1$.

    \vspace{0.3cm}

    Similarly, when $p(t) = 1$, $y_\beta = e^{-t + C_\beta}$. \\
    As such, `gluing' the functions together when $t = 1$, we find that $y_\alpha(t) = e^{-2}$ and $y_\beta(t) = e^{-1 + C_\beta}$ so $C_\beta = -1$ and the equation is solved.
\end{exercise}


\end{document}
