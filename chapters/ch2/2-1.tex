\documentclass[../../diff_eqs.tex]{subfiles}

\begin{document}

% docs/syntax:

% definitions
% \begin{definition}[Definition]
%     Definition 1
% \end{definition}

% hr
% \hr

% exercise
% \begin{exercise}{problem number}
%
%    problem starts
% \end{exercise}
%%%%%%%%%%%%%%%%%%%%%%%%%%%%%%%%%%%%%%%

% this is pdf page 38 // 24 in the actual textbook.

If $\frac{dy}{dt} = f(t, y)$ and $f$ is linear (w.r.t $y$), then we can rewrite it in the following form (called the \textbf{first-order linear differential equation}):

$$ \frac{dy}{dt} + p(t)y = g(t) \Longleftrightarrow P(t)\frac{dy}{dt} + Q(t)y = G(t) \text{ (page 24)}$$ 

\begin{definition}[Integrating Factor]
    A \textbf{integrating factor} $\mu(t)$ is a function such that when a diffy q is multiplied by it, the equation is then immediately integratable (discovered by Leibniz). (page 25)
\end{definition}

\begin{exercise}{- Pauls Online Notes, Problem 4 (modified)}

    Find the general solution to the ODE $$t\frac{dy}{dt} + 2y = t^2 - t + 1\text{.}$$

    \hr

    This diffy q looks hard. To start, we add on an integrating factor $\alpha(t)$ to the equation to get 
    
    $$t \alpha(t) \frac{dy}{dt} + 2 \alpha(t) y = \alpha(t)(t^2 - t + 1)\text{.}$$

    From here, consider what happens when you take the derivative of $(t \cdot y \cdot \alpha(t))$:\footnote{This is not the \textit{actual} way to do it \textendash Paul's math notes first divides everything by $t$ so they only have to consider the derivative of $y \alpha(t)$.}

    $$\frac{d}{dt} \left[t \cdot y \cdot \alpha(t)\right] = y\alpha(t) + t \alpha(t) \frac{dy}{dt} + ty \alpha'(t) = t \alpha(t) \frac{dy}{dt} + y(\alpha(t) + t\alpha'(t))\text{.}$$

    For this equation to match the left hand side of the equation above, we then must have that $t\alpha(t) = t\alpha(t)$ and $2\alpha(t) = \alpha(t) + t \alpha'(t) \rightarrow \alpha(t) = t \alpha'(t)$. From that last equation, I recognized that the function $\alpha(t) = t$ works!
    
    And from there, after plugging things in and integrating, I ended up with my final answer that the general solution to the given ODE is $$y(t) = \boxed{\frac{t^2}{4} - \frac{t}{3} + \frac{1}{2} + \frac{C}{t^2}}\text{.}$$

\end{exercise}

So essentially, the process of solving diffy qs of the form $P(t)\frac{dy}{dt} + Q(t)y = G(t)$ is to first divide by $P(t)$, then find an integrating factor that ``matches up'' both sides of the equation. 

Mathematically:

$$ P(t)\frac{dy}{dt} + Q(t)y = G(t) \rightarrow \frac{dy}{dt} + \frac{Q(t)}{P(t)} y = \frac{G(t)}{P(t)} \rightarrow \kappa(t)\frac{dy}{dt} + \kappa(t) \frac{Q(t)}{P(t)}y = \kappa(t)\frac{G(t)}{P(t)}$$

\vspace{12pt}

Since $(y \kappa(t))' = \kappa(t) \cdot y' + \kappa'(t) y$, comparing terms on the LHS, we get that we just need to find some $\kappa(t)$ such that $\kappa'(t) = \kappa(t)\frac{Q(t)}{P(t)}$. If that nasty fraction $\left(\frac{Q(t)}{P(t)}\right)$ is some constant or basic polynomial, the equation is \textit{probably} solvable. 

So assuming that some suitable $\kappa(t)$ is found, we then just kinda evaluate everything from there.
$$ \longrightarrow \int \frac{d}{dt} \left[y \kappa(t)\right] = \int \kappa(t)\frac{G(t)}{P(t)} \Longrightarrow y(t) = \frac{C + \int \kappa(t)\frac{G(t)}{P(t)}}{\kappa(t)}\text{.}$$

It's quite messy when written out.

(27) For equations of the form $\mathlarger{\frac{dy}{dt} + ay = g(t)}$, the right integrating factor is $\mu(t) = e^{at}$. This can be rederived pretty easily (probably). \\

If you want to integrate but the messy thing doesn't simplify, that's fine; put the bounds of your integral to be from some arbitrary $t_0$ to $t$, preferably in a way such that if an initial condition $y(y_0) = c_0$, $t_0 = y_0$. In this way, your integral will collapse on itself when evaluated at $y = y_0$ and any other value of your function will be computed by that constant given in the problem plus the accumulated gain/loss from the function as it goes to your desired $x$/$y$ value.

%%%%%%%%%%%% exercises %%%%%%%%, page 31

\end{document}
