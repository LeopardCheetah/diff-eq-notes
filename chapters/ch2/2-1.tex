\documentclass[../../diff_eqs.tex]{subfiles}

\begin{document}

% docs/syntax:

% definitions
% \begin{definition}[Definition]
%     Definition 1
% \end{definition}

% hr
% \hr

% exercise
% \begin{exercise}{problem number}
%
%    problem starts
% \end{exercise}
%%%%%%%%%%%%%%%%%%%%%%%%%%%%%%%%%%%%%%%

% this is pdf page 38 // 24 in the actual textbook.

If $\frac{dy}{dt} = f(t, y)$ and $f$ is linear (w.r.t $y$), then we can rewrite it in the following form (called the \textbf{first-order linear differential equation}):

$$ \frac{dy}{dt} + p(t)y = g(t) \Longleftrightarrow P(t)\frac{dy}{dt} + Q(t)y = G(t) \text{ (page 24)}$$ 

\begin{definition}[Integrating Factor]
    A \textbf{integrating factor} $\mu(t)$ is a function such that when a diffy q is multiplied by it, the equation is then immediately integratable (discovered by Leibniz). (page 25)
\end{definition}

\begin{exercise}{- Pauls Online Notes, Problem 4 (modified)}

    Find the general solution to the ODE $$t\frac{dy}{dt} + 2y = t^2 - t + 1\text{.}$$

    \hr

    This diffy q looks hard. To start, we add on an integrating factor $\alpha(t)$ to the equation to get 
    
    $$t \alpha(t) \frac{dy}{dt} + 2 \alpha(t) y = \alpha(t)(t^2 - t + 1)\text{.}$$

    From here, consider what happens when you take the derivative of $(t \cdot y \cdot \alpha(t))$:\footnote{This is not the \textit{actual} way to do it \textendash Paul's math notes first divides everything by $t$ so they only have to consider the derivative of $y \alpha(t)$.}

    $$\frac{d}{dt} \left[t \cdot y \cdot \alpha(t)\right] = y\alpha(t) + t \alpha(t) \frac{dy}{dt} + ty \alpha'(t) = t \alpha(t) \frac{dy}{dt} + y(\alpha(t) + t\alpha'(t))\text{.}$$

    For this equation to match the left hand side of the equation above, we then must have that $t\alpha(t) = t\alpha(t)$ and $2\alpha(t) = \alpha(t) + t \alpha'(t) \rightarrow \alpha(t) = t \alpha'(t)$. From that last equation, I recognized that the function $\alpha(t) = t$ works!
    
    And from there, after plugging things in and integrating, I ended up with my final answer that the general solution to the given ODE is $$y(t) = \boxed{\frac{t^2}{4} - \frac{t}{3} + \frac{1}{2} + \frac{C}{t^2}}\text{.}$$

\end{exercise}

So essentially, the process of solving diffy qs of the form $P(t)\frac{dy}{dt} + Q(t)y = G(t)$ is to first divide by $P(t)$, then find an integrating factor that ``matches up'' both sides of the equation. 

Mathematically:

$$ P(t)\frac{dy}{dt} + Q(t)y = G(t) \rightarrow \frac{dy}{dt} + \frac{Q(t)}{P(t)} y = \frac{G(t)}{P(t)} \rightarrow \kappa(t)\frac{dy}{dt} + \kappa(t) \frac{Q(t)}{P(t)}y = \kappa(t)\frac{G(t)}{P(t)}$$

\vspace{12pt}

Since $(y \kappa(t))' = \kappa(t) \cdot y' + \kappa'(t) y$, comparing terms on the LHS, we get that we just need to find some $\kappa(t)$ such that $\kappa'(t) = \kappa(t)\frac{Q(t)}{P(t)}$. If that nasty fraction $\left(\frac{Q(t)}{P(t)}\right)$ is some constant or basic polynomial, the equation is \textit{probably} solvable. 

So assuming that some suitable $\kappa(t)$ is found, we then just kinda evaluate everything from there.
$$ \longrightarrow \int \frac{d}{dt} \left[y \kappa(t)\right] = \int \kappa(t)\frac{G(t)}{P(t)} \Longrightarrow y(t) = \frac{C + \int \kappa(t)\frac{G(t)}{P(t)}}{\kappa(t)}\text{.}$$

It's quite messy when written out.

(27) For equations of the form $\mathlarger{\frac{dy}{dt} + ay = g(t)}$, the right integrating factor is $\mu(t) = e^{at}$. This can be rederived pretty easily (probably). 

\vspace{0.3cm}

If you want to integrate but the messy thing doesn't simplify, that's fine; put the bounds of your integral to be from some arbitrary $t_0$ to $t$, preferably in a way such that if an initial condition $y(y_0) = c_0$, $t_0 = y_0$. In this way, your integral will collapse on itself when evaluated at $y = y_0$ and any other value of your function will be computed by that constant given in the problem plus the accumulated gain/loss from the function as it goes to your desired $x$/$y$ value.

%%%%%%%%%%%% exercises %%%%%%%%, page 31

% exercise
\begin{exercise}{1c-8c}
    
    (I'm just going to document my answers here.) \\
    1c: $\mathlarger{y(t) = e^{-2t} + \frac{1}{3}\left(t - \frac{1}{3}\right) + \frac{C}{e^{3t}}}$. \\
    2c: $\mathlarger{y(t) = \left(\frac{1}{3}t^3 + C\right)e^{2t}}$. \\
    3c: $\mathlarger{y(t) = \frac{t^2}{2e^t} + 1 + \frac{C}{e^t}}$. \\
    4c: $\mathlarger{y(t) = 1.5\sin(2t) + \frac{0.75}{t}\cos(2t) + \frac{C}{t}}$. \\ 
    5c: $\mathlarger{y(t) = -3e^t + Ce^{2t}}$. \\
    6c: $\mathlarger{y(t) = -te^{-t} + Ct}$. \\ 
    7c: $\mathlarger{y(t) = \sin(2t) - 2\cos(2t) + \frac{C}{e^t}}$. Warning: This integral is hard but is very doable. \\ 
    8c: $\mathlarger{y(t) = 3t^2 - 12t + 24 + \frac{C}{2e^{t/2}}}$. 
\end{exercise}


\begin{exercise}{9-12}

    (More answer exercise documentation. Both the general form and the specific solution to each problem will be given.) \\ 
    9: $\mathlarger{y(t) = 2te^{2t} - 2e^{2t} + Ce^t}$. The specific case when $y(0) = 1$ is given by $C = 3$. \\
    10: $\mathlarger{y(t) = \frac{t^2/2 + C}{e^{2t}}}$. The specific solution when $y(1) = 0$ is given by $C = -\frac{1}{2}$. \\
    11: $\mathlarger{y(t) = \frac{\sin t + C}{t^2}}$, with $C = 0$. \\ 
    12: $\mathlarger{y(t) = \frac{t - 1}{t} + \frac{C}{te^t}}$, $C = 2$ for this particular case.
\end{exercise}



\begin{exercise}{18}

    Given the simplicity of the right hand side, we can fake solve the equation for $y(t)$; namely, from intuition, if $y(t) = \alpha t + \beta$, then $y'(t) = \alpha$ (a constant) and we can probably find values $\alpha$, $\beta$ that make such a solution possible.

    In fact we do; $y(t) = -\frac{3}{4}t + \frac{21}{8} + Ce^{-2t/3}$, where that last term was derived from realizing that if we actually integrated this properly, our integrating factor $\mu(t)$ would be $e^{2t/3}$. 
    
    Anyways, things get a little dicey from here. Let's call the point where $y(t)$ touches (but doesn't cross) the $t$-axis as $t_0$. Then, we know that $y'(t_0) = 0$ ($y$ must be at a local max/min as otherwise $y$ would cross the $t$-axis) and $y(t_0) = 0$. From here, we can rearrange our equations as follows: 

    $$y'(t_0) = 0 \rightarrow 0 = -\frac{3}{4} - \frac{2}{3}Ce^{-2t_0/3} \rightarrow -\frac{9}{8} = Ce^{-2t_0/3}\text{ and}$$
    $$y(t_0) = 0 \longrightarrow \frac{3}{4}t_0 - \frac{21}{8} = Ce^{-2t_0/3}$$ 
    and we can match the LHSs of both equations to get $\frac{3}{4}t_0 - \frac{21}{8} = -\frac{9}{8}$ and find that $t_0 = 2$. 
    
    From here, we can simply plug this value of $t_0$ into our equations and solve for $C$:

    $$y'(t_0) = 0 \rightarrow y'(2) = 0 \rightarrow C = -\frac{9}{8}e^{4/3}$$

    Thus, $y_0 = y(0) = \frac{21}{8} + C = \frac{21}{8} - \frac{9}{8}e^{4/3} \approx \boxed{-1.64}$.
\end{exercise}


\begin{exercise}{20}

    (Note: this problem and the last problem have caused me some amount of pain because I keep misreading the problem and not sticking to the end.)

    If you want $y' - y = 1 + 3 \sin t$ to remain finite as $t \to \infty$, then when you get that $y(t) = - 1 - \frac{3}{2}\left(\cos t + \sin t\right) + Ce^t$, it should be pretty clear that $C = 0$. As such, $y_0 = y(0) = -1-\frac{3}{2}(1 + 0) = \boxed{-\frac{5}{2}}.$

    When doing this problem, don't doubt yourself :).
\end{exercise}


\begin{exercise}{28 - Variation of Parameters}
    
    (a). If $g(t) = 0 \ \forall t$, then effectively $g(t) = 0$. As such, we are simply solving $\frac{dy}{dt} + p(t)y = 0$ which can be done by separating variables: 

    $$\frac{dy}{dt} = -p(t) \cdot y \rightarrow \frac{1}{y} dy = -p(t) dt \rightarrow \ln(y) = \int -p(t) dt + C_0 \text{ so } y(t) = C_1\exp\left(-\int p(t) dt\right)$$ 
    where $C_1 = e^{C_0}$. Replace $C_1$ with $A$ to get the expression shown in the textbook.

    \vspace{0.3cm}

    (b). To show $A(t)$ must satisfy (51), we simply substitute everything in and cancel the messy equation. 
    $$y' + p(t)y = g(t) \Longrightarrow \left[A(t)\exp\left(- \int p(t) dt\right)\right]' + A(t)\exp\left(- \int p(t) dt\right) p(t) = g(t)$$
    $$\longrightarrow \exp\left(-\int p(t) dt \right)\left([A'(t) + A(t)(-p(t))] + A(t)p(t)\right) = g(t) \text{ so } A'(t) = g(t) \cdot \exp\left(\int p(t) dt \right)$$
    which is exactly the equation given by (51).

    \vspace{0.3cm}

    (c). This part is lowkey quite simple. Picking up from (b), we simply slap an integral sign in front of the massive equation that we derived for $A'(t)$, and after replacing $A(t)$ with the appropriate integral in an integral, it is equivalent to (33) up to a constant as the $\mu(t)$ in (33) is really the big scary integral we've been dealing with, $\int p(t) dt$.
\end{exercise}

\end{document}
