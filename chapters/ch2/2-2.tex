\documentclass[../../diff_eqs.tex]{subfiles}

\begin{document}

% docs/syntax:

% definitions
% \begin{definition}[Definition]
%     Definition 1
% \end{definition}

% hr
% \hr

% exercise
% \begin{exercise}{problem number}
%
%    problem starts
% \end{exercise}
%%%%%%%%%%%%%%%%%%%%%%%%%%%%%%%%%%%%%%%

%% textbook page 33.

A general first-order differential equation can be written as $\mathlarger{\frac{dy}{dx} = f(x, y)}$ which can be rearranged to become $\mathlarger{M(x, y) + N(x, y)\frac{dy}{dx} = 0}$. When $M$ is a function solely of $x$ and $N$ is a function solely of $y$, then we can rewrite the diffy q as
$$M(x) dx + N(y) dy = 0$$

which we call a \textbf{separable} equation. (A definition of a separable differential equation is a differential equation that can be written as the form above.)

To solve these equations, it's just `simple'/`intuitive' integration. I'm not sure what exactly to write so.....

Let $A'(x) = a(x)$ and $B'(y) = b(y)$. Then, 
$$a(x) + b(y)\frac{dy}{dx} = 0$$
simlifies down to 
$$A(x) + B(y) = c$$ 
for an arbitrary constant $c$. 

%% on to problems! page 38.
\begin{exercise}{1-4}

    (Solutions only.) \\ 
    1: $\mathlarger{\frac{y^2}{2} = \frac{x^3}{3} + C}$ or $\mathlarger{y = \sqrt{\frac{2}{3}x^3 + C}}$. \\ 
    2: $\mathlarger{y = \frac{1}{C - \cos x}}$. \\ 
    3: $\mathlarger{\frac{1}{2}\tan(2y) = \frac{x}{2} + \frac{\sin(2x)}{4} + C}$. Alternatively, the textbook gives the clever solution of $\mathlarger{y = \frac{\pi}{2}}$ (a constant) and similar solutions as $y' = 0$ and the RHS evaluates to 0. Very tricky (but nice) stuff. \\ 
    4: $\mathlarger{\ln |x| + C = \arcsin y}$. Didn't fully get those inverse trig integrals right away :/. 

    (The rest (5-8) look trivial as you just move terms to either side of the equation and integrate.)
\end{exercise}

\begin{exercise}{17}

    Literally just separate and integrate. 
    $$ \Rightarrow \int 3y^2 - 6y \ dy = \int 1 + 3x^2 \ dx + C \rightarrow y^3 - 3y^2 = x + x^3 + C$$

    Plugging in our initial condition of $y(0) = 1$ (e.g. the values $x = 0$ and $y = 1$), we get that $C = 1 - 3 = -2$. 

    To determine the interval in which the solution is valid, we simply look at when the denominator of $y'$ is 0 \textemdash $3y^2 - 6y = 0 \rightarrow y = 0, 2$ which correspond to the $x$ values of $0$, $\frac{-1 + \sqrt{3}}{2}$, $\frac{-1 - \sqrt{3}}{2}$, and $x = -1$ ($y(-1) = 2$ and $y$ of any of the other $x$ values is equal to 0).

    I'm not sure how really to proceed from here but I think the way to actually do it is to recognize that $y(-1) = 2$, $y(1) = 0$, which given our initial condition $y(0) = 1$ means that our function $y$ is trapped between $-1$ and $1$ ($y$ is not exactly a ``function'' at these specific points so our domain then becomes $(-1, 1)$).

    It's not a great solution :/.    
\end{exercise}

\begin{exercise}{19}

    $$\frac{dy}{dx} = 2y^2 + xy^2 \rightarrow \int \frac{1}{y^2} \ dy = \int 2 + x \ dx \rightarrow y = \frac{1}{2x + \frac{x^2}{2} + C}\text{.}$$
    
    Plugging our initial condition $y(0) = 1$ nets $C = -1$.
    
    To find the minimum value of $y$, we simply find the derivative $y'$ and set it to 0;
    $$y' = \frac{1}{\left(2x + \frac{x^2}{2} - 1\right)} \cdot (2 + x)\text{.}$$

    Setting $y'$ to 0, the only solution we get is $x = -2$ so the minimum value of our function $y$ is $\mathlarger{y(-2) = -\frac{1}{-4 + 2 - 1} = \boxed{\frac{1}{3}}}$.

    While it is true that $y(3) < y(-2)$, note that the function $y$ is discontinuous and plotting the graph on Desmos, we see that $y(3)$ is not `on' the particular branch of the solution we're focused on (namely, the piece of the function where $y(0) = 1$). 
\end{exercise}


\end{document}
