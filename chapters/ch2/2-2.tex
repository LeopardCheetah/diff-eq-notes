\documentclass[../../diff_eqs.tex]{subfiles}

\begin{document}

% docs/syntax:

% definitions
% \begin{definition}[Definition]
%     Definition 1
% \end{definition}

% hr
% \hr

% exercise
% \begin{exercise}{problem number}
%
%    problem starts
% \end{exercise}
%%%%%%%%%%%%%%%%%%%%%%%%%%%%%%%%%%%%%%%

%% textbook page 33.

A general first-order differential equation can be written as $\mathlarger{\frac{dy}{dx} = f(x, y)}$ which can be rearranged to become $\mathlarger{M(x, y) + N(x, y)\frac{dy}{dx} = 0}$. When $M$ is a function solely of $x$ and $N$ is a function solely of $y$, then we can rewrite the diffy q as
$$M(x) dx + N(y) dy = 0$$

which we call a \textbf{separable} equation. (A definition of a separable differential equation is a differential equation that can be written as the form above.)

To solve these equations, it's just `simple'/`intuitive' integration. I'm not sure what exactly to write so.....

Let $A'(x) = a(x)$ and $B'(y) = b(y)$. Then, 
$$a(x) + b(y)\frac{dy}{dx} = 0$$
simlifies down to 
$$A(x) + B(y) = c$$ 
for an arbitrary constant $c$. 

%% on to problems! page 38.
\begin{exercise}{1-4}

    (Solutions only.) \\ 
    1: $\mathlarger{\frac{y^2}{2} = \frac{x^3}{3} + C}$ or $\mathlarger{y = \sqrt{\frac{2}{3}x^3 + C}}$. \\ 
    2: $\mathlarger{y = \frac{1}{C - \cos x}}$. \\ 
    3: $\mathlarger{\frac{1}{2}\tan(2y) = \frac{x}{2} + \frac{\sin(2x)}{4} + C}$. Alternatively, the textbook gives the clever solution of $\mathlarger{y = \frac{\pi}{2}}$ (a constant) and similar solutions as $y' = 0$ and the RHS evaluates to 0. Very tricky (but nice) stuff. \\ 
    4: $\mathlarger{\ln |x| + C = \arcsin y}$. Didn't fully get those inverse trig integrals right away :/. 

    (The rest (5-8) look trivial as you just move terms to either side of the equation and integrate.)
\end{exercise}

% ill finish the rest later

\end{document}
