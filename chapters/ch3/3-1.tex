\documentclass[../../diff_eqs.tex]{subfiles}

\begin{document}

%%%%%%%%%%%%%%%%%%%%%%%%%%%%%%%%%%%%%%%%%%%%%%%%%%%%%%%%%%%%
% docs/syntax:

% definitions
% \begin{definition}[Definition]
%     Definition 1
% \end{definition}

% hr
% \hr

% exercise
% \begin{exercise}{problem number}
%
%    problem starts
% \end{exercise}

%%%%%%%%%%%%%%%%%%%%%%%%%%%%%%%%%%%%%%%%%%%%%%%%%%%%%%%%%%%%

Remember, a \textbf{linear} second-order differential equation is of the form 
$$P(t)y'' + Q(t)y' + R(t)y = G(t)\text{.}$$

\textbf{Nonlinear} differential equations are super hard and annoying to tackle and as such they're just not tackled in this book :/.

In second-order differential equations, a problem with an initial condition has initial condition of the form 

$$\begin{cases}
y(t_0) = y_0 \\ 
y'(t_0) = y_0'
\end{cases}\ \text{.}$$

Note that there are two initial equations given - the location of $y$ at time $t_0$, and the slope of $y$ at time $t_0$.


\begin{definition}[Homogenous]
    A \textbf{homogenous} differential equation has no `constant' terms (terms without $y$). In the case for our second-order linear differential equations, a homogenous equation of that form can be written as 

    $$P(t)y'' + Q(t)y' + R(t)y = 0\text{.}$$    
\end{definition}

Anyways, it turns out if we solve the homogenous version of the differential equation $P(t)y'' + Q(t)y' + R(t)y = G(t)$, we can actually find an expression for $y$ (that may or may not have an integral in it). That's pretty cool.

\vspace{0.3cm}

For this chapter (unfortunately), we will only consider the cases when $\boxed{P, \ Q, \text{ and } R \text{ are \textbf{constants}}}$.

\vspace{0.3cm}

Thus, our differential equation becomes $ay'' + by' + cy = 0$. Letting $y = e^{rt}$, we find that our equation now becomes 

$$\rightarrow ar^2e^{rt} + bre^{rt} + ce^{rt} = e^{rt}(ar^2 + br + c) = 0$$

with $ar^2 + br + c$ called the \textbf{characteristic equation} for the general differential equation with constant coefficients shown above.

\vspace{0.3cm}

If we let $r_1$ and $r_2$ be two real roots that satisfy the characteristic equation above, then the \textbf{general solution} to our differential equation is $y = c_1e^{r_1t} + c_2e^{r_2t}$ with $c_1$ and $c_2$ being arbitrary constants. Initial conditions can be solved for summarily.



\begin{exercise}{1-4}

    1. $y = c_1e^t + c_2e^{-3t}$. \\
    2. $y = c_1e^{t} + c_2e^{2t}$. \\ 
    3. $y = c_1e^{t/2} + c_2e^{-t/3}$. \\
    4. $y = c_1 + c_2e^{-5t}$. 
\end{exercise}

\begin{exercise}{13}
    
    If a differential equation's solution is $\mathlarger{c_1e^2t + c_2e^{-3t}}$, we have $r_1 = 2$, $r_2 = -3$ and as such our differential equation is $y'' + y' - 6y = 0$.

    It probably can be shown that no other differential equation produces the general solution given in the problem.
\end{exercise}

\begin{exercise}{16}

    The characteristic equation for our differential equation is $n^2 - n - 2 = 0$ and as such we have roots $r_1 = 2$, $r_2 = -1$. As such, the general solution to the equation is $y = c_1e^{2t} + c_2e^{-t}$. 

    To make the solution approach $0$ as $t \to \infty$, we need $c_1 = 0$ as in any other case, $e^{2t}$ will spiral out to infinity and our solution is unbounded. Thus, we can plug this solution into the second part of the initial value problem $y'(0) = 2$:

    $$y'(0) = 2 \rightarrow 2 = 2\cdot 0e^{2t} + (-1)\cdot c_2e^{-0} \rightarrow c_2 = -2\text{.}$$

    Thus, our final solution to the differential equation is $y_{sol} = -2e^{-t}$ and $y_{sol}(0) = \alpha = -2$. 
\end{exercise}



\end{document}
