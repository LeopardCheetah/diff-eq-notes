\documentclass[../../diff_eqs.tex]{subfiles}

\begin{document}

%%%%%%%%%%%%%%%%%%%%%%%%%%%%%%%%%%%%%%%%%%%%%%%%%%%%%%%%%%%%
% docs/syntax:

% definitions
% \begin{definition}[Definition]
%     Definition 1
% \end{definition}

% hr
% \hr

% exercise
% \begin{exercise}{problem number}
%
%    problem starts
% \end{exercise}

%%%%%%%%%%%%%%%%%%%%%%%%%%%%%%%%%%%%%%%%%%%%%%%%%%%%%%%%%%%%


What happens when the roots of the characteristic equation $ar^2 + br + c = 0$ for a general differential equation $ay'' + by' + cy = 0$ are imaginary?

\vspace{0.2cm}

Let the roots $r_1$ and $r_2$ of the characteristic equation be $r_1 = \alpha + i\beta$ and $r_2 = \alpha - i\beta$ for real $\alpha$, $\beta$. Then, the corresponding solutions to the differential equation are 

$$\begin{cases}
    y_1 = e^{(\alpha + i \beta)t} = e^{\alpha t}\cos(\beta t) + ie^{\alpha t}\sin(\beta t) \text{ and} \\
    y_2 = e^{(\alpha - i \beta)t} = e^{\alpha t}\cos(\beta t) - ie^{\alpha t}\sin(\beta t) 
\end{cases} \; \text{.}$$

In Section 3.2 (Thereom 3.2.6), it was mention that the real and imaginary parts of any solution to a given differential equation are each solutions to the given differential equation. In our case thus, $y_3 = e^{\alpha t}\cos(\beta t)$ and $y_4 = e^{\alpha t}\sin(\beta t)$ are also solutions to $ay'' + by' + cy = 0$, with $W[y_3, y_4] = \beta e^{2\alpha t} \not = 0$.

\vspace{0.2cm}
\hr
\vspace{0.2cm}

\begin{exercise}{6-8}

    6. The quadratic yields roots $r_1, r_2 = 1 \pm i\sqrt{5}$ so the corresponding general solution is $c_1e^t\cos(\sqrt{5}t) + c_2e^t\sin(\sqrt{5}t)$. \\ 
    7. $y = c_1e^{-t}\cos(t) + c_2e^{-t}\sin(t)$. \\ 
    8. $y = c_1e^{-3t}\cos(2t) - c_2e^{-3t}\sin(2t)$.
\end{exercise}

\begin{exercise}{25}

    (a): $\mathlarger{\frac{dy}{dt} = \frac{dy}{dx}\frac{dx}{dt} = \frac{1}{t}\frac{dy}{dx}}$. \\
    $\mathlarger{\frac{d^2y}{dt^2} = \frac{d}{dt}\left(\frac{1}{t} \cdot \frac{dy}{dx}\right) = -\frac{1}{t^2}\frac{dy}{dx} + \frac{dx}{dt}\frac{d}{dx}\left(\frac{dy}{dx}\right) = \frac{1}{t^2}\left(\frac{d^2y}{dx^2} - \frac{dy}{dx}\right)}$.

    \vspace{0.2cm}

    (b): Simplify substitute everything we just derived in into the equation .\_.
\end{exercise}

\begin{exercise}{26-29}
    
    As seen from question 25, we can transform the coefficients of the differential equation $(t^2, \alpha t, \beta)$ into $(1, \alpha - 1, \beta)$. In this case, $\alpha = 1$ and $\beta = 1$ so our new differential equation is 

    $$\frac{d^2y}{dx^2} + y = 0$$

    which has solutions $y_1 = \cos(x)$, $y_2 = \sin(x)$. As such, since $x = \ln t$, $y_1 = \cos(\ln t)$ and $y_2 = \sin(\ln t)$ are a set of solutions to the differential equation in terms of $t$.

    \hr

    27: $\alpha = 4$ and $\beta = 2$ so $y_1 = e^{-x} = \frac{1}{t}$ and $y_2 = e^{-2x} = \frac{1}{t^2}$.

    28: $y_1 = \frac{1}{t}$ and $y_2 = t^6$.
    
    29: $y_1 = t^2$ and $y_2 = t^3$. 
\end{exercise}




\end{document}
