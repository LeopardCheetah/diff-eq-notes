\documentclass[../../diff_eqs.tex]{subfiles}

\begin{document}

%%%%%%%%%%%%%%%%%%%%%%%%%%%%%%%%%%%%%%%%%%%%%%%%%%%%%%%%%%%%
% docs/syntax:

% definitions
% \begin{definition}[Definition]
%     Definition 1
% \end{definition}

% hr
% \hr

% exercise
% \begin{exercise}{problem number}
%
%    problem starts
% \end{exercise}

%%%%%%%%%%%%%%%%%%%%%%%%%%%%%%%%%%%%%%%%%%%%%%%%%%%%%%%%%%%%

Thank you Lagrange for this method.

Lagrange's idea to solving general differential equations $L[y] = g(t)$ is to replace constants with functions:

Say we have a differential equation $y'' + p(t)y' + q(t)y = g(t)$ and we know the complementary solution $y_c(t) = c_1y_1 + c_2y_2$ to the homogenous version of the differential equation. From here, the idea is to replace the constants $c_1$ and $c_2$ with functions $u_1$ and $u_2$ so $y = u_1y_1 + u_2y_2$ ends up being a particular solution to the differential equation.

Assuming this, we differentiate our particular solution:

$$\longrightarrow y' = u'_1 y_1 + u_1y'_1 + u'_2y_2 + u_2y'_2\text{.}$$

Since we're not interested in solving another second-order differential equation and we have a free condition we can impose on the equation, we let $u'_1 y_1 + u'_2y_2 = 0$ so that we have 
$$y' = u_1y'_1 + u_2y'_2\text{.}$$

As such, differentiating again, we have 

$$y'' = u'_1 y'_1 + u_1 y''_1 + u'_2 y'_2 + u_2 y''_2\text{.}$$

From here, substituting in $y''$ and $y'$ and $y$ into the general differential equation, much simplifcation eventually leads us to find $u'_1 y'_1 + u'_2 y'_2 = g(t)$.

Thus, with this equation, we have a linear system from which we can solve for $u_1$ and $u_2$:

$$\begin{cases}
    u'_1 y'_1 + u'_2 y'_2 = g(t) \text{ (derived)} \\ 
    u'_1 y_1 + u'_2y_2 = 0 \text{ (mandated \textendash \ see above)}
\end{cases}\text{.}$$

\vspace{0.25cm}

The solutions to this system ends up being 
$$
\mathlarger{
\begin{cases}
    u_1 = -\int \frac{y_2g}{W[y_1, y_2]} \ dt + c_1 \\ 
    u_2 = \int \frac{y_1g}{W[y_1, y_2]} \ dt + c_2
\end{cases}
}
$$

with $W[a, b](t) = a(t)b'(t) - a'(t)b(t)$. Thanks Lagrange :).

Note that this methodology is not a silver bullet \textemdash \ $y_1$ and $y_2$ may be hard to find solutions for if $p(t)$ and $q(t)$ are complicated, and the integrals solving for $u_1$ and $u_2$ may vary in nice-ness to solve.

\end{document}
