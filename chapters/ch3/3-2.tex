\documentclass[../../diff_eqs.tex]{subfiles}

\begin{document}

%%%%%%%%%%%%%%%%%%%%%%%%%%%%%%%%%%%%%%%%%%%%%%%%%%%%%%%%%%%%
% docs/syntax:

% definitions
% \begin{definition}[Definition]
%     Definition 1
% \end{definition}

% hr
% \hr

% exercise
% \begin{exercise}{problem number}
%
%    problem starts
% \end{exercise}

%%%%%%%%%%%%%%%%%%%%%%%%%%%%%%%%%%%%%%%%%%%%%%%%%%%%%%%%%%%%

\begin{definition}[Differential Operator L]
    A general differential operator \textit{does stuff}. 
    
    For now, for continuous functions $\alpha$ and $\beta$ on some open interval $I$ and for any function $\phi$ twice differentiable on $I$, we define the \textbf{differential operator L} as 

    $$L[\phi] = \phi'' + \alpha\phi' + \beta\phi\text{.}$$    

    Note that the result of applying $L$ to some function $f$ is another function $g$.
\end{definition}

In this section we wil examine the equation $L[y] = 0$.

\vspace{0.3cm}

\begin{definition}[Existence and Uniqueness Theorem]
    (Reproduced from page 110.) \\
    Consider the initial value problem

    $$y'' + p(t)y' + q(t)y = g(t), \ \ y(t_0) = y_0, \ y'(t_0) = y_0'\text{,}$$

    where $p$, $q$, and $g$ are continuous on an open interval $I$ with $t_0 \in I$. This problem has exactly one solution $y = \phi(t)$, and the solution exists throughout the interval $I$.    
\end{definition}
This existence theorem is pretty similar to Theorem 2.4.1 but generalized to second-order linear differential equations. Note once again the guarantee and uniqueness of a solution to the given differential equation over a certain interval.


\vspace{0.3cm}

\begin{definition}[Principle of Superposition]
    If $y_1$ and $y_2$ are two solutions to the differential equation $L[y] = 0$, then $c_1y_1 + c_2y_2$ is also a solution to the given differential equation for any $(c_1, c_2) \in \mathbb{R}^2$.
\end{definition}

\vspace{0.3cm}
\begin{definition}{Wronskian Determinant}
    The \textbf{Wronskian Determinant} for the system 
    $$\begin{cases}
        c_1y_1(t_0) + c_2y_2(t_0) = y_0, \\ 
        c_1y_1'(t_0) + c_2y_2'(t_0) = y_0' 
    \end{cases}$$

    is 
    $$W = \begin{vmatrix}
        y_1(t_0) & y_2(t_0) \\ y_1'(t_0) & y_2'(t_0)
    \end{vmatrix} = y_1(t_0)y_2'(t_0) - y_1'(t_0)y_2(t_0)\text{.}$$
    
    If $W$ is non-zero, then there is a unique solution to the differential equation $L[y] = 0$ with \textbf{any} given initial condition. Otherwise, there are initial conditions to the differential equation that cannot be satisfied no matter how $c_1$ and $c_2$ are chosen (113).
\end{definition}


Note that if the Wronskian $W$ is non-zero, the two solutions $y_1$ and $y_2$ to $L[y] = 0$ are said to form a \textbf{fundamental set of solutions}.

(There's a lot more discussion here about uniqueness of solutions, Wronksians, and other things I frankly don't care about.)


\vspace{0.3cm}
Regarding complex valued solutions, if $y = u(t) + iv(t)$ satisfies $L[y] = 0$, then $u$ and $v$ are also solutions to the differential equation $L[y] = 0$ (Theorem 3.2.6, Page 117). This is pretty important and notable!


\vspace{0.3cm}
For another theorem in this long section, we have....
\begin{definition}[Abel's Theorem]
    
\end{definition}

\end{document}
