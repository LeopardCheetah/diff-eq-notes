\documentclass[../../diff_eqs.tex]{subfiles}

\begin{document}

%%%%%%%%%%%%%%%%%%%%%%%%%%%%%%%%%%%%%%%%%%%%%%%%%%%%%%%%%%%%
% docs/syntax:

% definitions
% \begin{definition}[Definition]
%     Definition 1
% \end{definition}

% hr
% \hr

% exercise
% \begin{exercise}{problem number}
%
%    problem starts
% \end{exercise}

%%%%%%%%%%%%%%%%%%%%%%%%%%%%%%%%%%%%%%%%%%%%%%%%%%%%%%%%%%%%

An improper integral is an integral that has an infinity in one of its bounds or has its function be otherwise discontinuous. Examples of discontinuous integrals are 

$$\int_a^{\infty} f(t) \ dt \text{ or } \int_0^5 \frac{1}{x - 3} \ dx \text{.}$$

Sometimes these integrals converge, sometimes they diverge. 

\begin{definition}[Piecewise Continuous]
    We call a function $f$ \textbf{piecewise continuous} over an interval $\alpha \leq t \leq \beta$ if we can find a finite number of points $\alpha = t_0 < t_1 < \cdots < t_n = \beta$ such that 
    \begin{enumerate}
        \item $f$ is continuous on each open subinterval $t_{i - 1} < t < t_i$, and
        \item Approaching $f$'s endpoints of each subinterval from within the subinterval results in a finite limit.
    \end{enumerate}
    
    As the book sums it up nicely, $f$ is piecewise ocntinuous if it is ``continuous except for a finite number of jump discontinuities.''


    The integral of a piecewise function $f$ is just the sum of its parts:

    $$\int_{\alpha}^{\beta} f(t) \ dt = \sum_{i = 0}^{n - 1} \int_{t_i}^{t_{i + 1}} f(t) \ dt = \int_{\alpha}^{t_1} f(t) \ dt + \int_{t_1}^{t_2} f(t) \ dt + \cdots + \int_{t_{n - 1}}^{\beta} f(t) \ dt \text{.}$$
\end{definition}


With that, we get our first theorem on divergence and convergence:

\begin{definition}[Integral Convergence (Theorem 6.1.1)]
    If $f$ is piecewise continuous for $t \geq \alpha$ and $|f(t)| \leq g(t)$ when $t > C$ for some constant $C$, if $\mathlarger{\int_C^{\infty} g(t) \ dt }$ converges, then $\mathlarger{\int_a^{\infty} f(t) \ dt }$ also converges.

    Moreover, if $f(t) \geq g(t) \geq 0$ for all $t \geq C$, if $\mathlarger{\int_C^{\infty} g(t) \ dt }$ diverges, then $\mathlarger{\int_a^{\infty} f(t) \ dt }$ also diverges.
\end{definition}

Essentially, this theorem states that if $f$ is bounded above and the bounding function is convergent, $f$ is similarly convergent. Also, if $f$ is bounded below and the integral of the bounding function diverges, then $f$ is similarly divergent.


\subsubsection{The Laplace Transform}

\begin{definition}[Integral Transform]
    An integral transform is a relation of the form 

    $$F(s) = \int_{\alpha}^{\beta} K(s, t)f(t) \ dt$$

    where $K$ is called the \textbf{kernel} of the transformation and limits $\alpha$ and $\beta$ are given. $F$ is called the \textbf{transform} of $f$.
\end{definition}

In our case, the laplace transform for a given function $f$ is defined as 

$$\mathcal{L}\{f(t)\} = F(s) = \int_0^{\infty} e^{-st} f(t) \ dt$$

if this improper integral converges. Thus, the kernel function used here is $K(s, t) = e^{-st}$.

\vspace{0.2cm}

In general, to solve a differential equation, we use a laplace transform on a function $f$ to derive $F$, solve for $F$, then un-transform and recover $f$. In general, $s$ may also be a complex number.

\begin{definition}[Laplace Transform Existence (Theorem 6.1.2)]
    If $f$ is piecewise continuous for all $t \geq 0$ and there exist constnats $K > 0$, $a$, and $M > 0$ such that 

    $$\left|f(t)\right| \leq Ke^{at} \text{ when } t \geq M\text{,}$$

    then the laplace transform $\mathcal{L}\{f\} = F$ as defined above exists for all $s > a$.
\end{definition}

If a function $f$ satisfies Theorem 6.1.2, then $f$ is described as being piecewise continuous and of \textbf{exponential order} as $t \to \infty$ (e.g. The highest `order' the function can be is $O(e^{at})$).

\vspace{0.2cm}

// (Some examples of the Laplace transformation being used out in the wild are given!!) // 


Note that the Laplace transform is a \textbf{linear operator}; that is, 

$$\mathcal{L}\{c_1f_1(t) + c_2f_2(t)\} = c_1\mathcal{L}\{f_1(t)\} + c_2\mathcal{L}\{f_2(t)\}\text{.}$$


\begin{exercise}{6-11}

    The laplace transform of $\mathlarger{e^{\alpha t} = \frac{1}{s - b}}$, assuming $s > b > 0$. Thus, for (6), 

    $$\mathcal{L}\{\cosh(bt)\} = \mathcal{L}\left\{\frac{e^{bt}}{2} + \frac{e^{-bt}}{2}\right\} = \frac{1}{2}\mathcal{L}\{e^{bt}\} + \frac{1}{2}\mathcal{L}\{e^{-bt}\} = \frac{1}{2(s - b)} + \frac{1}{2(s + b)}\text{.}$$

    The rest of the exercises similarly follow; 

    $$\text{(7): } \mathcal{L}\{\sinh(bt)\} = \frac{1}{2}\mathcal{L}\{e^{bt}\} - \frac{1}{2}\mathcal{L}\{e^{-bt}\} = \frac{1}{2(s - b)} - \frac{1}{2(s + b)}\text{.}$$

    $$\text{(8): } \L{\sin(bt)} = \frac{1}{2i}\L{e^{ibt}} - \frac{1}{2i}\L{e^{-ibt}} = \frac{1}{2i(s - ib)} - \frac{1}{2i(s + ib)} = \frac{(s + ib) - (s - ib)}{2i(s - ib)(s + ib)} = \frac{b}{s^2 + b^2}\text{.}$$

    $$\text{(9): } \L{\cos(bt)} = \frac{1}{2}\L{e^{ibt}} + \frac{1}{2}\L{e^{-ibt}} = \frac{1}{2(s - ib)} + \frac{1}{2(s + ib)}\text{.}$$

    $$\text{(10): } \L{e^{at}\sin(bt)} = \frac{1}{2i}\L{e^{t(a + ib)}} - \frac{1}{2i}\L{e^{t(a - ib)}} = \frac{1}{2i((s - a) - ib)} - \frac{1}{2i((s - a) + ib)}$$ 
    $$= \frac{(s - a + ib) - (s - a - ib)}{2i((s - a) + ib)((s - a) - ib)} = \frac{b}{b^2 + (s - a)^2}\text{.}$$

    $$\text{(11): } \L{e^{at}\cos(bt)} = \frac{1}{2}\L{e^{t(a + ib)}} + \frac{1}{2}\L{e^{t(a - ib)}} = \frac{1}{2(s - a - ib)} + \frac{1}{2(s - a + ib)}$$
    $$= \frac{s - a + ib + s - a - ib}{2(s - a - ib)(s - a + ib)} = \frac{s - a}{(s - a)^2 + b^2}\text{.}$$
\end{exercise}


\begin{exercise}{4}

    Recall from above that $\mathlarger{\mathcal{L}\{f\} = \int_0^{\infty}e^{-st} f \ dt }$. Now, to find $\mathcal{L}\{t^{n}\}$, assume we know $\mathcal{L}\{t^{n - 1}\} = g(s)$ $(t > 1)$.

    Then, by integration by parts,

    $$\mathcal{L}\{t^n\} = \int_0^{\infty} e^{-st}t^n \ dt = \left[\frac{t^n}{s}e^{-st}\right]_0^{\infty} + \frac{n}{s} \int_0^{\infty} e^{-st}t^{n - 1}\text{.}$$

    The second expression simply simplifies to $\mathlarger{\frac{n}{s}g(s)}$ by definition of the laplace transform, and the first expression is 0 as when plugging in the upper bound $t = 0$, $t^n = 0$, and when plugging the lower bound $t = \infty$, $e^{st}$ outgrows $t^n$ (assuming $s > 0$) so $\mathlarger{\frac{1}{s} \cdot \frac{t^n}{e^{st}} = 0}$ as $t \to \infty$.
\end{exercise}

\begin{exercise}{5}
    
    $$\mathcal{L}\{\cos(at)\} = I = \int_0^{\infty} e^{-st}\cos(at) \ dt \Rightarrow \left[\frac{1}{s}e^{-st}\cos(at)\right]_0^{\infty} - \frac{a}{s}\int_0^{\infty} e^{-st} \sin(at) \ dt$$
    $$= \frac{1}{s} - \frac{a}{s}\left(\left[\frac{1}{s}e^{-st}\sin(at)\right]_0^{\infty} + \frac{a}{s}\int_0^{\infty} e^{-st}\cos(at)\right) = \frac{1}{s} - \frac{a}{s}\left(\frac{a}{s}I\right) \rightarrow \boxed{I = \frac{1}{s\left(1 + \frac{a^2}{s^2}\right)}}\text{.}$$
\end{exercise}

% do 16-18, 24

\begin{exercise}{16-18}
    
    \vspace{0.2cm}

    16. $\mathlarger{\L{f} = \int_0^{\infty} e^{-st}f \ dt = \int_0^{\pi} e^{-st} \ dt = \frac{1}{s}(1 - e^{-s\pi}) }$.

    \vspace{0.2cm}

    17. $\mathlarger{\L{f} = \int_0^1 e^{-st}t \ dt + \int_1^{\infty} e^{-st} \ dt = \left[-\frac{1}{s}e^{-st}\right]_1^{\infty} + \left[\frac{t}{s}e^{-st}\right]_1^0 + \frac{1}{s}\int_0^1 e^{-st} \ dt = \frac{1 - e^{-s}}{s^2}}$.

    \vspace{0.2cm}

    18. 
    $$ \L{f} = \int_0^1 e^{-st}t \ dt + \int_1^2 e^{-st}(2 - t) \ dt = \left(-\frac{e^{-s}}{s} + \frac{1}{s^2} - \frac{e^{-s}}{s^2}\right) + 2 \int_1^2 e^{-st} \ dt - \int_1^2 e^{-st}t \ dt $$
    $$ = \left(-\frac{e^{-s}}{s} + \frac{1}{s^2} - \frac{e^{-s}}{s^2}\right) + \frac{2e^{-s}}{s} - \frac{2e^{-2s}}{s} - \left(\left[-\frac{t}{s}e^{-st}\right]_1^2 + \frac{1}{s} \int_1^2 e^{-st} \ dt \right) $$ 
    
    $$= -\frac{e^{-s}}{s} + \frac{1}{s^2} - \frac{e^{-s}}{s^2} + \frac{2e^{-s}}{s} - \frac{2e^{-2s}}{s} - \frac{e^{-s}}{s} + \frac{2}{s}e^{-2s} - \frac{e^{-s}}{s^2} + \frac{e^{-2s}}{s^2} = \boxed{\frac{1 - 2e^{-s} + e^{-2s}}{s^2}}\text{.}$$
\end{exercise}

% do 24
\begin{exercise}{24}

    24(ab): Making the substitution $x = st$ and $dx = s \ dt$ (and assuming $s > 0$ else the integral bounds must be switched), 

    $$\L{t^p} = \int_0^{\infty} e^{-x} \left(\frac{x}{s}\right)^p \ \frac{dx}{s} = \frac{1}{s^{p + 1}} \int_0^{\infty} e^{-x} x^p \ dx$$

    which is exactly what needs to be shown. By definition, the integral we have in the simplified form of $\L{t^p}$ evaluates to $\Gamma(p + 1)$\footnote{the textbook question for this part of the question is wrong :/}, so 

    $$\L{t^p} = \frac{\Gamma{p+1}}{s^{p + 1}} = \frac{n!}{s^{n + 1}}$$

    if $n \in \mathbb{N}$. 
    
    \vspace{0.2cm}

    24(c): We essentially make the substitution $x = y^2$ ($\sqrt{x} = y$, $dx = 2y \ dy$):

    $$\L{t^{-1/2}} = \frac{1}{\sqrt{s}} = \int_0^{\infty} e^{-x} \frac{1}{\sqrt{x}} \ dx \ \longrightarrow \ \frac{1}{\sqrt{s}} \int_{\sqrt{0} = 0}^{\sqrt{\infty} = \infty} e^{-y^2} \frac{1}{y} \ 2y \ dy = \frac{2}{\sqrt{s}} \int_0^{\infty} e^{-y^2} \ dy $$ 

    which is exactly what's asked for.

    \vspace{0.2cm}

    24(d): 
    $$\L{t^{1/2}} = \frac{1}{s\sqrt{s}} \int_0^{\infty} e^{-x} \sqrt{x} \ dx \rightarrow \frac{1}{s \sqrt{s}} \int_0^{\infty} (y)\left(2ye^{-y^2}\right) \ dy = \frac{1}{s\sqrt{s}}\left(\left[-ye^{-y^2}\right]_0^{\infty} + \int_0^{\infty} e^{-y^2} \ dy\right) = \boxed{\frac{\sqrt{\pi}}{2s\sqrt{s}}}\text{.}$$
\end{exercise}


\end{document}
